% Options for packages loaded elsewhere
\PassOptionsToPackage{unicode}{hyperref}
\PassOptionsToPackage{hyphens}{url}
\PassOptionsToPackage{dvipsnames,svgnames,x11names}{xcolor}
%
\documentclass[
]{article}

\usepackage{amsmath,amssymb}
\usepackage{iftex}
\ifPDFTeX
  \usepackage[T1]{fontenc}
  \usepackage[utf8]{inputenc}
  \usepackage{textcomp} % provide euro and other symbols
\else % if luatex or xetex
  \usepackage{unicode-math}
  \defaultfontfeatures{Scale=MatchLowercase}
  \defaultfontfeatures[\rmfamily]{Ligatures=TeX,Scale=1}
\fi
\usepackage{lmodern}
\ifPDFTeX\else  
    % xetex/luatex font selection
\fi
% Use upquote if available, for straight quotes in verbatim environments
\IfFileExists{upquote.sty}{\usepackage{upquote}}{}
\IfFileExists{microtype.sty}{% use microtype if available
  \usepackage[]{microtype}
  \UseMicrotypeSet[protrusion]{basicmath} % disable protrusion for tt fonts
}{}
\makeatletter
\@ifundefined{KOMAClassName}{% if non-KOMA class
  \IfFileExists{parskip.sty}{%
    \usepackage{parskip}
  }{% else
    \setlength{\parindent}{0pt}
    \setlength{\parskip}{6pt plus 2pt minus 1pt}}
}{% if KOMA class
  \KOMAoptions{parskip=half}}
\makeatother
\usepackage{xcolor}
\ifLuaTeX
  \usepackage{luacolor}
  \usepackage[soul]{lua-ul}
\else
  \usepackage{soul}
  
\fi
\setlength{\emergencystretch}{3em} % prevent overfull lines
\setcounter{secnumdepth}{5}
% Make \paragraph and \subparagraph free-standing
\ifx\paragraph\undefined\else
  \let\oldparagraph\paragraph
  \renewcommand{\paragraph}[1]{\oldparagraph{#1}\mbox{}}
\fi
\ifx\subparagraph\undefined\else
  \let\oldsubparagraph\subparagraph
  \renewcommand{\subparagraph}[1]{\oldsubparagraph{#1}\mbox{}}
\fi


\providecommand{\tightlist}{%
  \setlength{\itemsep}{0pt}\setlength{\parskip}{0pt}}\usepackage{longtable,booktabs,array}
\usepackage{calc} % for calculating minipage widths
% Correct order of tables after \paragraph or \subparagraph
\usepackage{etoolbox}
\makeatletter
\patchcmd\longtable{\par}{\if@noskipsec\mbox{}\fi\par}{}{}
\makeatother
% Allow footnotes in longtable head/foot
\IfFileExists{footnotehyper.sty}{\usepackage{footnotehyper}}{\usepackage{footnote}}
\makesavenoteenv{longtable}
\usepackage{graphicx}
\makeatletter
\def\maxwidth{\ifdim\Gin@nat@width>\linewidth\linewidth\else\Gin@nat@width\fi}
\def\maxheight{\ifdim\Gin@nat@height>\textheight\textheight\else\Gin@nat@height\fi}
\makeatother
% Scale images if necessary, so that they will not overflow the page
% margins by default, and it is still possible to overwrite the defaults
% using explicit options in \includegraphics[width, height, ...]{}
\setkeys{Gin}{width=\maxwidth,height=\maxheight,keepaspectratio}
% Set default figure placement to htbp
\makeatletter
\def\fps@figure{htbp}
\makeatother
% definitions for citeproc citations
\NewDocumentCommand\citeproctext{}{}
\NewDocumentCommand\citeproc{mm}{%
  \begingroup\def\citeproctext{#2}\cite{#1}\endgroup}
\makeatletter
 % allow citations to break across lines
 \let\@cite@ofmt\@firstofone
 % avoid brackets around text for \cite:
 \def\@biblabel#1{}
 \def\@cite#1#2{{#1\if@tempswa , #2\fi}}
\makeatother
\newlength{\cslhangindent}
\setlength{\cslhangindent}{1.5em}
\newlength{\csllabelwidth}
\setlength{\csllabelwidth}{3em}
\newenvironment{CSLReferences}[2] % #1 hanging-indent, #2 entry-spacing
 {\begin{list}{}{%
  \setlength{\itemindent}{0pt}
  \setlength{\leftmargin}{0pt}
  \setlength{\parsep}{0pt}
  % turn on hanging indent if param 1 is 1
  \ifodd #1
   \setlength{\leftmargin}{\cslhangindent}
   \setlength{\itemindent}{-1\cslhangindent}
  \fi
  % set entry spacing
  \setlength{\itemsep}{#2\baselineskip}}}
 {\end{list}}
\usepackage{calc}
\newcommand{\CSLBlock}[1]{\hfill\break\parbox[t]{\linewidth}{\strut\ignorespaces#1\strut}}
\newcommand{\CSLLeftMargin}[1]{\parbox[t]{\csllabelwidth}{\strut#1\strut}}
\newcommand{\CSLRightInline}[1]{\parbox[t]{\linewidth - \csllabelwidth}{\strut#1\strut}}
\newcommand{\CSLIndent}[1]{\hspace{\cslhangindent}#1}

\makeatletter
\@ifpackageloaded{tcolorbox}{}{\usepackage[skins,breakable]{tcolorbox}}
\@ifpackageloaded{fontawesome5}{}{\usepackage{fontawesome5}}
\definecolor{quarto-callout-color}{HTML}{909090}
\definecolor{quarto-callout-note-color}{HTML}{0758E5}
\definecolor{quarto-callout-important-color}{HTML}{CC1914}
\definecolor{quarto-callout-warning-color}{HTML}{EB9113}
\definecolor{quarto-callout-tip-color}{HTML}{00A047}
\definecolor{quarto-callout-caution-color}{HTML}{FC5300}
\definecolor{quarto-callout-color-frame}{HTML}{acacac}
\definecolor{quarto-callout-note-color-frame}{HTML}{4582ec}
\definecolor{quarto-callout-important-color-frame}{HTML}{d9534f}
\definecolor{quarto-callout-warning-color-frame}{HTML}{f0ad4e}
\definecolor{quarto-callout-tip-color-frame}{HTML}{02b875}
\definecolor{quarto-callout-caution-color-frame}{HTML}{fd7e14}
\makeatother
\makeatletter
\@ifpackageloaded{caption}{}{\usepackage{caption}}
\AtBeginDocument{%
\ifdefined\contentsname
  \renewcommand*\contentsname{Table of contents}
\else
  \newcommand\contentsname{Table of contents}
\fi
\ifdefined\listfigurename
  \renewcommand*\listfigurename{List of Figures}
\else
  \newcommand\listfigurename{List of Figures}
\fi
\ifdefined\listtablename
  \renewcommand*\listtablename{List of Tables}
\else
  \newcommand\listtablename{List of Tables}
\fi
\ifdefined\figurename
  \renewcommand*\figurename{Figure}
\else
  \newcommand\figurename{Figure}
\fi
\ifdefined\tablename
  \renewcommand*\tablename{Table}
\else
  \newcommand\tablename{Table}
\fi
}
\@ifpackageloaded{float}{}{\usepackage{float}}
\floatstyle{ruled}
\@ifundefined{c@chapter}{\newfloat{codelisting}{h}{lop}}{\newfloat{codelisting}{h}{lop}[chapter]}
\floatname{codelisting}{Listing}
\newcommand*\listoflistings{\listof{codelisting}{List of Listings}}
\makeatother
\makeatletter
\makeatother
\makeatletter
\@ifpackageloaded{caption}{}{\usepackage{caption}}
\@ifpackageloaded{subcaption}{}{\usepackage{subcaption}}
\makeatother
\ifLuaTeX
  \usepackage{selnolig}  % disable illegal ligatures
\fi
\usepackage{bookmark}

\IfFileExists{xurl.sty}{\usepackage{xurl}}{} % add URL line breaks if available
\urlstyle{same} % disable monospaced font for URLs
\hypersetup{
  pdftitle={Unveiling the Complexity of Food Webs: A Comprehensive Overview of Definitions, Scales, and Mechanisms},
  pdfauthor={Tanya Strydom; Jennifer A. Dunne; Timothée Poisot; Andrew P. Beckerman},
  pdfkeywords={food web, network construction, scientific ignorance},
  colorlinks=true,
  linkcolor={blue},
  filecolor={Maroon},
  citecolor={Blue},
  urlcolor={Blue},
  pdfcreator={LaTeX via pandoc}}


\title{Unveiling the Complexity of Food Webs: A Comprehensive Overview
of Definitions, Scales, and Mechanisms}
\author{Tanya Strydom %
%
\textsuperscript{%
%
1%
}%
; Jennifer A. Dunne %
%
\textsuperscript{%
%
2%
}%
; Timothée Poisot %
%
\textsuperscript{%
3,%
4%
}%
; Andrew P. Beckerman %
%
\textsuperscript{%
%
1%
}%
}
\date{2024-09-25}

\usepackage{setspace}
\usepackage[left]{lineno}
\usepackage[letterpaper]{geometry}

\usepackage[nolists,noheads,markers]{endfloat}
\geometry{margin=2.5cm}

\begin{document}

\thispagestyle{empty}
{\bfseries\sffamily\Large Unveiling the Complexity of Food Webs: A
Comprehensive Overview of Definitions, Scales, and Mechanisms}
\vfil
Tanya Strydom %
%
\textsuperscript{%
%
1%
}%
; Jennifer A. Dunne %
%
\textsuperscript{%
%
2%
}%
; Timothée Poisot %
%
\textsuperscript{%
3,%
4%
}%
; Andrew P. Beckerman %
%
\textsuperscript{%
%
1%
}%

\vfil
{\small
\textbf{Abstract:} Food webs are a useful abstraction and representation
of the feeding links between species in a community and are used to
infer many ecosystem level processes. However, the different theories,
mechanisms, and criteria that underpin how a food web is defined and,
ultimately, constructed means that not all food webs are representing
the same ecological process. Here we present a synthesis of the
different assumptions, scales and mechanisms that are used to define
different ecological networks ranging from metawebs (an inventory of all
potential interactions) to fully realised networks (interactions that
occur within a given community over a certain timescale). Illuminating
the assumptions, scales, and mechanisms of network inference allows a
formal categorisation of how to use networks to answer key ecological
and conservation questions and defines guidelines to prevent
unintentional misuse or misinterpretation.
\vfil
\textbf{Keywords:} %
food web, network construction, %
scientific ignorance%
}
\clearpage
\setcounter{page}{1}
\doublespacing
\linenumbers

At the heart of modern biodiversity science are a set of concepts and
theories about biodiversity, stability and function. These relate to the
abundance, distribution and services that biodiversity provides, and how
biodiversity -- as an interconnected set of species -- responds to
multiple stressors. The interaction between species (or individuals) is
one of the fundamental building blocks of ecological communities provide
a powerful abstraction that can help quantify, conceptualise, and
understand biodiversity dynamics, and ultimately, one hopes, make
prediction, mitigate change and manage services {[}ref{]}. Such network
representations of biodiversity (including within species diversity) are
increasingly argued to be an asset to predictive ecology, climate change
mitigation and resource management. Here, it is argued that
characterising biodiversity in a network will allow deeper capacity to
understand and predict the abundance, distribution, dynamics and
services provided by multiple species facing multiple stressors.

However, the way that a network is constructed (encoded) defines an
epistemology of the network concept which, we argue, can influence the
resulting observations and conclusions about pattern and mechanisms that
are made (Brimacombe et al., 2023; Proulx et al., 2005). This process of
constructing networks has two major pillars: the data and theory, the
latter representing an expression of mechanism and process giving rise
to patterns that emerge from collating interactions among species. Each
of these pillars carries with it a set of practical, semantic and
conceptual constraints that not only influence progress in making
network ecology more valuable and potentially predictive, but help
define the spatial, temporal and evolutionary scale of assumptions we
make and predictions we might generate from the networks.

With respect to data, it is extremely challenging to actually record
species interactions in the field (Jordano, 2016a, 2016b). Despite
notable herculean efforts (\textbf{Woodward? Benguela?} Maiorano et al.
(2020)), actual coverage of `real world' interaction data remains sparse
(Poisot et al., 2021). Against this practical challenge, there is
additionally high variance in the terminology we use to define networks.
Finally, the mathematical and statistical tools we use to construct,
conceptualise, analyse and predict with these networks are also highly
variable.

\begin{enumerate}
\def\labelenumi{\arabic{enumi}.}
\tightlist
\item
  what are the underlying assumptions about nodes, edges, scale and
  process that are made when we attempt to delimit and describe a food
  webs;
\item
  are there families of commonly used tools that map onto assumptions
  about scales and processes;
\end{enumerate}

The provision of this detail ultimately leads to a set of insights and
conclusions about whether, when and under what conditions network
representations of biodiversity can contribute to the advancement of
ecological theory and generate value in predictive ecology.
Specifically, we finish this perspective with an overview of fundamental
questions in ecology that we think can benefit from network thinking and
a proposal that such thinking can accelerate our capacity to predict the
impact of multiple stressors on biodiverse communities.

\section{Setting the Scene: The Not So Basics of Nodes and
Edges}\label{sec-anatomy}

Defining a food web seems simple; it is the representation of the
interactions (edges) between species (nodes), however the definition of
`edges' and `nodes', as well as the scale at which they are aggregated
can take many forms (Poisot, Stouffer, et al., 2016). An awareness of
variance in the way a food web can be defined is critical as it
represents the `object' that is used to make inferences either about the
interactions between species, or how the structure influences ecosystem
level processes. One thus needs to be aware of both the criteria that is
used to define nodes and edges, and what processes or mechanisms the
aggregation of the two represents, as this will ultimately determine and
delimit the way in which a network should be used.

\subsubsection{How do we define a node?}\label{how-do-we-define-a-node}

Although this may seem an elementary question in the context of food
webs --- a node \emph{should} represent a (taxonomic) species, the
reality is that nodes can often represent an aggregation of different
species - so called `trophic species' or segregation of species by life
stages. Representing nodes as non-taxonomic species can be useful in
certain contexts (Williams \& Martinez, 2000; Yodzis, 1982) and in cases
where the adult and larval stages of a species have different diets it
may make ecological sense (Clegg et al., 2018) meaning that it is not
uncommon that networks often have nodes that have different definitions
of a `species' \emph{e.g.} consisting of both taxonomic and trophic
species. Practical implications of how we are aggregating the nodes is
that the resolution may not always be `pixel perfect' \emph{i.e.,} we
may be unable to assess the co-extinction risk of a species pair,
however there is value in having nodes that represent an aggregation of
species, as these convey a much more general overview of how the links
are distributed within the community.

\subsubsection{What is meant by an
edge?}\label{what-is-meant-by-an-edge}

At its core, links within food webs can be thought of as a
representation of either feeding links between species - be that
realised (Pringle, 2020) or potential (Dunne, 2006), or representative
of fluxes within the community/system \emph{e.g.,} energy transfer or
material flow (Lindeman, 1942). How we specify links will influence the
resulting structure of the network - and the inferences we will make
thereof. For example taking a food web that consists of links
representing all \emph{potential} feeding links for a community
(\emph{i.e.,} a metaweb) will be meaningless if one is interested in
understanding the flow of energy through the network as the links within
a metaweb do not represent environmental/energetic constraints. In
addition to the various ways of defining the links between species pairs
there are also a myriad of ways in which the links themselves can be
quantified. Links between species are often treated as being present or
absent (\emph{i.e.,} binary) but it is also possible to use
probabilities (Banville et al., 2024; which quantifies how likely an
interaction is to occur, Poisot, Cirtwill, et al., 2016) or continuous
measurements (which quantifies the strength of of an interaction, Berlow
et al., 2004).

\subsubsection{Network representations}\label{network-representations}

Broadly, networks can be thought of to fall into two different `types';
namely metawebs; traditionally defined as all of the \emph{potential}
interactions for a specific species pool (Dunne, 2006), and realised
networks; which is the subset of interactions in a metaweb that are
\emph{realised} `on the ground'. The fundamental difference between
these two different types of networks is that a metaweb provides insight
as to the viability of an interaction between two species occurring and
is a means to identify links that are not ecologically plausible,
\emph{i.e.,} forbidden links (Jordano, 2016b), or an idea of the
\emph{complete} diet of a species (Strydom et al., 2023). Although
metawebs are typically `constrained' to a collection of species that
also co-occur, there is no reason that a metaweb can include species
that do not co-occur (although this would require some degree of
prediction/assumption to identify those possible interactions). In
contrast realised networks are highly localised and contingent on both
the co-occurrence of species as well as the influence of the
environment, and population and community dynamics on predator choice.
In the context of definitions and semantics the links that are
represented by a metaweb and a realised network are different; links
that are absent in a metaweb can be treated as being truly absent,
however links that are absent in a realised network cannot be considered
to be truly absent but are rather as absent due to the broader
environmental/community context. Importantly, a realised network is
\emph{not} simply the downscaling of a metaweb to a smaller scale
(\emph{e.g.,} moving from the country to the 1x1 km\textsuperscript{2}
scale based on fine-scale species co-occurrence) but represents a shift
towards capturing the higher level processes that determine the
\emph{realisation} of an interaction. Thus, metawebs and realised
networks are determined and constrained by a different set of
assumptions as to what processes are determining the presence/absence of
an interaction between two species as well as the resulting network
structure.

\section{From Nodes and Edges to Scales, Context, and
Processes}\label{sec-mechanisms}

Armed with these basics, it is now possible to review the scales and
assumptions that are made by a wide range of tools to assist in
constructing networks against poor data with the hope of capturing
important processes that underpin accurate prediction. Our thesis
centres on a five-tier conceptualization of networks: evolutionary
compatibility, co-occurrence, feasibility, abundance, predator choice,
and non-trophic interactions. In the following sections we review each
of these and then provide a synthesis among them.

\subsection{Understanding the processes that determine species
interactions}\label{sec-process}

Processes that are all-or-nothing (possibility) vs processes that are
context dependent (likelihood). Processes form the underlying logic of
models (and \st{arguably}, no, for sure even empirical data). This means
also the interplay of the two, \emph{i.e.,} the use of models to `gap
fill' within existing empirical dataset (Biton et al., 2024; Stock,
2021). Ultimately when we put this all together it will influence how we
can (and should) use the resulting network. Here we present
Figure~\ref{fig-feasibility} some of the processes that have been shown
to influence either/or the feasibility (possibility) of an interaction
occurring between two species or if a feasible interaction is realised
(likelihood of realisation) within the specific environmental/community
context. Of course these processes do not function in a vacuum and do
interact with/influence one another but it is still beneficial to
present them as such as these are often the underlying processes that
influence model development, the criteria for data collection in the
field, and the scale of organisation for which they are relevant
(species, population, community).

\begin{figure}

\centering{

\includegraphics{images/concept_v2.png}

}

\caption{\label{fig-feasibility}TODO.}

\end{figure}%

\textbf{Evolutionary compatibility}

There is compelling evidence that the possibility of an interaction
occurring between two species is the result of their shared
(co)evolutionary history (Dalla Riva \& Stouffer, 2016; Gómez et al.,
2010; Segar et al., 2020). In the more proximal sense this is manifested
as the `trait complementarity' between two species, whereby one species
(the predator) has the `correct' set of traits that allow it to chase,
capture, kill, and consume the other species (the prey). For species
pairs where this condition is not met the link is deemed to be forbidden
(Jordano, 2016b); \emph{i.e.,} not physically possible and will always
be absent within the network. In the context of trying to determine the
feasibility (\emph{i.e.,} the \emph{possibility}) of an interaction,
phylogeny is an excellent predictor (Fricke et al., 2022; Strydom et
al., 2022) and allows one to construct what can be considered to be a
metaweb. In terms of thinking about the anatomy of an `feasibility
network' one should be aware that it is possible to represent
interactions as either binary (feasible/forbidden; \emph{i.e.,} the
traditional definition of a metaweb Dunne (2006)) or as a probability
(Banville et al., 2024), where the probability represents how likely
that the interaction between to species is feasible (what is the
possibility of this interaction occurring?).

\textbf{(Co)occurrence}

Although the outright assumption that because two species are
co-occurring it must mean that they are interacting is inherently flawed
(Blanchet et al., 2020), it is of course impossible for two species to
interact (at least in terms of feeding links) if they are not
co-occurring in time and space. Thus co-occurrence data alone is
insufficient to build an accurate and ecologically meaningful
representation of a food web having information on the co-occurrence of
species can further aid us in refining metawebs by allowing us to
downsample the network based on the species found in a specific
location, or even add additional uncertainty based in how likely species
are to co-occur (\textbf{dansereauSpatiallyExplicitPredictions2023?}).
Additionally the interplay between the interaction between a species
pair and their co-occurrence is meaningful when one is operating in the
space of trying to determine the distribution of a species (Higino et
al., 2023), and forms a key component of some of the next generation
species distribution models \emph{e.g.,} joint SDMs (Pollock et al.,
2014).

\textbf{Abundance}

The abundance of the different species within the community can
influence the likelihood of an interaction occurring in a myriad of
ways. There is the argument that networks (and the interactions that
make them up) are driven by only the abundance of the different species
and not the characteristics (traits), \emph{sensu} neutral processes and
have been formalised with the neutral model (Canard et al., 2012), as
well as statistical tools (Momal et al., 2020). Alternatively the
abundance of species in a community can influence which interactions are
ultimately realised (Banville et al., 2024; Poisot et al., 2015).

\textbf{Predator choice (energetic cost)}

Ultimately, predator choice is underpinned by the energetic cost-benefit
of trying to catch, kill, and consume prey, and is well described within
optimal foraging theory {[}ref{]} and rests on the idea that the prey a
predator chooses to target is one that will have the greatest return on
energy with the lowest energetic cost. There are additional bodies of
work that attempt to include the cost of movement that the environment
imposes on an individual (Cherif et al., 2024) as well as 2D/3D search
space (Pawar et al., 2012). In terms of formalising these processes in
the context of predicting networks using diet models (Beckerman et al.,
2006; Petchey et al., 2008) that have predator choice determined by the
handling time, energy content, prey density, and predator attack rate.
Wootton et al. (2023) developed a model that moves the energy of the
system into different modules related to the process of the predator
acquiring energy from the prey \emph{i.e.,} compartmentation in food
webs (Krause et al., 2003).

\textbf{Indirect interactions}

The realisation (presence/absence) or strength of trophic interactions
themselves can also be modified by other, indirect (non-trophic),
interactions (Golubski \& Abrams, 2011; Pilosof et al., 2017), this can
be either `directly' through \emph{e.g.,} competition or `indirectly'
\emph{e.g.,} mutualistic/facilitative interactions will alter the
fine-scale distribution and abundance of some species (Kéfi et al.,
2012, 2015).

It should be self evident that the different processes discussed above
are all ultimately going to influence the realisation of interactions as
well as the structure of a network, however they are acting at different
scales of organisation. Both the \textbf{co-occurrence} and the
\textbf{evolutionary compatibility} are valid at the scale of the
species pair of interest, that is the \emph{possibility} of an
interaction being present/absent is assessed at the pairwise level and
one is left with a `list' of interactions that are present/absent.
Although it is possible to build a network (\emph{i.e.,} metaweb) from
this information it is important to be aware that the structure of this
network is not constrained by real-world dynamics or conditions
(\emph{i.e.,} community context), just because species are able to
interact does not mean that they will (Poisot et al., 2015). In order to
construct a network who's structure is a closer approximation of reality
(localised interactions) one needs to take into consideration properties
of the community as a whole and not just the two species of interest.

\textbf{downsampling paragraph??}

\section{Network prediction is
nuanced}\label{network-prediction-is-nuanced}

The different models that are used to either predict or construct
networks have an underlying philosophy that often only captures one or a
few of the processes discussed in Section~\ref{sec-process}, has
implications for how the resulting network is defined
Section~\ref{sec-anatomy}, which will ultimately delimit and define what
inferences can be made from the resulting network. Selecting a model for
the task of network prediction should come down to two things; what
\emph{aspect} of a food web one is interested in predicting, and what
data are available, necessary, and sufficient, and what is the purpose
of wanting to predict a network? It is important that a researcher is
aware of this to ensure that the appropriate model is selected. Broadly
researchers will be interested in predicting/constructing two different
types of networks; \emph{metawebs}, which is essentially a list of all
interactions that are \emph{possible} for a specific community
(\emph{i.e.,} at the scale of the species pairs), or being able to
predict location specific, \emph{realised}, networks for the community
(\emph{i.e.,} at the scale of the community). The nature of metawebs
means that they are unable to capture the structural metrics of
realised/`real-world' networks (Caron et al., 2024). The researcher is
also constrained by the data needs of both the model as well as the
network type; for example in order to predict a realised network one
needs additional community/population level data (\emph{e.g.,}
abundance), making metawebs a more feasible choice in data-poor contexts
(\emph{e.g.,} Strydom et al. (2023) construct a metaweb using a species
list and a phylogenetic tree). The final question is assessing the
purpose of predicting a network - is it to create a series of simulated,
species agnostic but still ecologically plausible, networks
{[}\emph{e.g.,}{]} or to predict a network for a specific community at a
specific location. It is these three points that will ultimately dictate
which model is going to best allow one to predict the appropriate
network.

\subsection{How do we predict food
webs?}\label{how-do-we-predict-food-webs}

There as many ways to predict networks as what there is to define them
and along with taking into consideration the points raised in the
previous section it is also beneficial to think about the context in
which the different models were developed - and how this will influence
the networks that they produce\ldots{}

There is a bit of a `point of conflict' between those calling for `pixel
perfect', regional scale data (Pringle, 2020; Pringle \& Hutchinson,
2020) and for the means to generate networks that are ecologically
plausible \emph{representations} (\emph{sensu} structural networks).
This represents two challenges; one is that models that represent
generalisations of networks often lack the ability to retrieve any
species/community specificity which limits their utility for real world,
species-driven scenarios \emph{e.g.,} species driven conservation
efforts (Dunn et al., 2009), however networks that are constructed
through either (most) empirical observations or through predictive means
are fundamentally going to represent metawebs, \emph{i.e.,} lack
constrained links, a representation of structure, or energy flow\ldots{}

\begin{tcolorbox}[enhanced jigsaw, leftrule=.75mm, breakable, colback=white, toptitle=1mm, opacityback=0, bottomrule=.15mm, colframe=quarto-callout-note-color-frame, toprule=.15mm, arc=.35mm, title=\textcolor{quarto-callout-note-color}{\faInfo}\hspace{0.5em}{Box 1 - Why we need to aggregate networks at different scales: A
hypothetical case study}, bottomtitle=1mm, titlerule=0mm, colbacktitle=quarto-callout-note-color!10!white, opacitybacktitle=0.6, rightrule=.15mm, left=2mm, coltitle=black]

Although it might seem most prudent to be predicting, constructing, and
defining networks that are the closest representation of reality there
are pros and cons of constructing both realised networks as well as
metawebs. Let us take for example a community across time/through
seasons. In this community we expect species to be either present or
absent depending on the season (\emph{i.e.,} changes in co-occurrence)
as well as some species exhibiting seasonal diet shifts, these details
would be lost at the scale of the metaweb an it would be valuable to
construct either smaller metawebs for the different seasonal communities
(thereby capturing the changes in community diversity), or realised
networks for each season (to capture diet or ecosystem process shifts).
However, these small-scale networks lack the context of the bigger
picture that is available at the metaweb - that is it gives us a more
holistic idea of the entire diet range of a specific species, which is
important when one needs to make conservation-based/applied decisions
(\emph{e.g.,} conserving the entire diet of a species and not just
seasonal prey items) as well as providing information on interactions
that may be possible regardless of the environmental/community context
(species may have the capacity to consume certain prey items but do not
do so due to local conditions). With this is mind let us see how the
different network aggregations can be used

\textbf{1: A global metaweb}

Knowledge of the entire diet breadth of a species is valuable especially
in terms of understanding how a species will respond to changes in the
community - \emph{e.g.,} invasions/rewilding exercises (where does the
new species `fit' within the network?) as well as potential capacity to
shift its diet. ALthough this might make sense across space and not time
but certain species act as links across the landscape {[}Rooney{]}

\textbf{2: A seasonal metaweb}

Knowledge at the finer scale is also valuable to understand/identify
that there are in fact differences between the seasons

\textbf{3: A seasonal realised network}

Dynamics are useful because they are a representation of the different
configurations/energy flows/ecosystem processes. Also to detect more
nuanced shifts in diet - \emph{e.g.,} seasonal diet shifts.

\textbf{Data trade off}

Above we highlight the practical uses of the different network
configurations but we also need to take into consideration the barriers
to construction/associated data needs/cost and acknowledge them.
Basically in the ideal world we would have all this information at hand
but in reality we might be sitting with seasonal metawebs\ldots{}

\end{tcolorbox}

\subsubsection{Models that predict structure}\label{sec-network-build}

Although we identify mechanisms that determine species interactions in
Section~\ref{sec-process} not all models that are used to predict
networks operate at this `mechanistic' level (at least in absolute
terms), but rather represent the \emph{structure} of a network based on
a series of \emph{a priori} assumptions of network connectance
(\emph{e.g.,} the niche model Williams \& Martinez (2000); although see
Allesina \& Pascual (2009) for a parameter-free model) or other
structural features of a \emph{realised} network (\emph{e.g.,}
stochastic block model, Xie et al. (2017)). Importantly these structural
models do not make species specific predictions (they are usually
species agnostic and treat nodes as trophic species) and so cannot be
used to determine if an interaction is either possible \emph{or}
realised between two species (\emph{i.e.,}one cannot use these models to
determine if species \(a\) eats species \(b\)). Although this means this
suite of models are unsuitable as tools for predicting interactions,
they have been shown to be sufficient tools to predict the structure of
networks (Williams \& Martinez, 2008).

\subsubsection{Models that predict metawebs (feasible
interactions)}\label{models-that-predict-metawebs-feasible-interactions}

\subsubsection{Models that predict realised networks (realised
interactions)}\label{models-that-predict-realised-networks-realised-interactions}

\section{Making Progress with
Networks}\label{making-progress-with-networks}

\subsection{Further development of models and
tools}\label{further-development-of-models-and-tools}

As we show in \textbf{?@tbl-families} there has been a suite of models
that have been developed to predict trophic links, however we are
lacking in tools that are explicitly taking into consideration
estimating both the feasibility as well as realisation of links,
\emph{i.e.,} both interactions and structure simultaneously (Strydom et
al., 2021). This could be addressed either through the development of
tools that do both (predict both interactions and structure), or it
might be possible to do a ensemble modelling approach (Becker et al.,
2022). Alternatively the development of tools that will allow for the
downsampling of metawebs into realised networks (\emph{e.g.,}
Roopnarine, 2006), although deciding exactly what is driving differences
between local networks and the regional metaweb might not be that simple
(Saravia et al., 2022). Probably also something that aligns with trying
to predict interaction strength - because that would be the gold
standard. Probably also worth just plainly stating that feasibility of
developing a model that is both broadly generalisable, but also cas
local specificity is probably not attainable (Stouffer, 2019), and more
specifically the potential use un models untangling/identifying the
different processes (Song \& Levine, 2024)

\subsection{At what scale should we be predicting/using
networks?}\label{at-what-scale-should-we-be-predictingusing-networks}

Look at Hutchinson et al. (2019)

We lack a clear agenda (and conceptualisation) as to what the
appropriate level of aggregation is for a `network'. Realistically most
empirical networks are more aligned with metawebs as opposed to realised
networks as they are often the result of some sort of aggregation of
observations across time, this creates a two-fold problem. Firstly, we
need to think about how this affects any sort of development of theory
that sits closer to the `realised network' side of the spectrum - how
often are we trying to ask and answer questions about realised networks
using feasible networks? The second is that this lack of `direction' as
to how we should define a network is (actually) probably one of the
biggest barriers that is affecting the use of networks in applied
settings\ldots{} By define I mean both delimiting the time and
geographic scale at which a network is aggregated at (Estay et al.,
2023). This is important because it can influence the inferences made,
\emph{e.g.,} the large body of work (landscape theory for food web
architecture) that showcases how different species use the landscape
will influence network dynamics (Rooney et al., 2008). There is also a
bit of an interplay with time and data and the different scales that
they may be integrated at - co-occurrence may span decades and just
because two species have been recorded in teh same space does not mean
it was at the same timescale (Brimacombe et al., 2024)

\subsection{How should we use different
networks?}\label{how-should-we-use-different-networks}

What for and how we can use networks is perhaps one of the biggest
`gaps' we have in network ecology (Tim's EBV ms), and there is a serious
need to start drawing clear, ecological links between network form and
function (although see Delmas et al., 2019). That being said one of the
most important things we can do is to be aware of the parameter space
that is possible given a specific definition of a network and operate
within those parameters. And we should use this in how we also
evaluate/benchmark the performance of the different models as well;
Poisot (2023) presents a set of guidelines for assessing how well a
model recovers pairwise interactions but we lack any clear strategies
for benchmarking structure.

\subsection{Feasible, realised, or
sustainable?}\label{feasible-realised-or-sustainable}

When do we determine a link to be `real'\ldots{} In the context of
feasible networks this is perhaps clearer - if all things were equal
(\emph{i.e.,} community context is irrelevant) would the predator be
able to consume the prey. However in the realised space there is also
the question of the long term `energetic feasibility' of an interaction
- just because an interaction is possible in the now is it able to
sustain a population in the long term. And what is the scale for that
long term - are we thinking at the generational scale? Because
ultimately when we are constructing a network we are aggregating not
only across space but also across time\ldots{} This is probably again a
Lokta-Volterra space question and something that the dynamic foodweb
model (Curtsdotter et al., 2019; Delmas et al., 2017; Lajaaiti et al.,
2024) is addressing, but again it is integrating this with the
feasible/realised axis.

\section{Concluding remarks}\label{concluding-remarks}

I think a big take home will (hopefully) be how different approaches do
better in different situations and so you as an end user need to take
this into consideration and pick accordingly. I think Petchey et al.
(2011) might have (and share) some thoughts on this. I feel like I need
to look at Berlow et al. (2008) but maybe not exactly in this context
but vaguely adjacent. This is sort of the crux of the argument presented
in Brimacombe et al. (2024) as well.

Do we expect there to be differences when thinking about unipartite vs
bipartite networks? Is there underlying ecology/theory that would assume
that different mechanisms (and thus models) are relevant in these two
`systems'.

\begin{itemize}
\tightlist
\item
  The Terry \& Lewis (2020) paper looks at some methods but is
  specifically looking at a bipartite world\ldots{}
\end{itemize}

\section*{References}\label{references}
\addcontentsline{toc}{section}{References}

\phantomsection\label{refs}
\begin{CSLReferences}{1}{0}
\bibitem[\citeproctext]{ref-allesinaFoodWebModels2009}
Allesina, S., \& Pascual, M. (2009). Food web models: A plea for groups.
\emph{Ecology Letters}, \emph{12}(7), 652--662.
\url{https://doi.org/10.1111/j.1461-0248.2009.01321.x}

\bibitem[\citeproctext]{ref-banvilleDecipheringProbabilisticSpecies2024}
Banville, F., Strydom, T., Blyth, P., Brimacombe, C., Catchen, M. D.,
Dansereau, G., Higino, G., Malpas, T., Mayall, H., Norman, K., Gravel,
D., \& Poisot, T. (2024). \emph{Deciphering probabilistic species
interaction networks}. EcoEvoRxiv. \url{https://doi.org/10.32942/X28G8Z}

\bibitem[\citeproctext]{ref-beckerOptimisingPredictiveModels2022}
Becker, D. J., Albery, G. F., Sjodin, A. R., Poisot, T., Bergner, L. M.,
Chen, B., Cohen, L. E., Dallas, T. A., Eskew, E. A., Fagre, A. C.,
Farrell, M. J., Guth, S., Han, B. A., Simmons, N. B., Stock, M.,
Teeling, E. C., \& Carlson, C. J. (2022). Optimising predictive models
to prioritise viral discovery in zoonotic reservoirs. \emph{The Lancet
Microbe}, \emph{3}(8), e625--e637.
\url{https://doi.org/10.1016/S2666-5247(21)00245-7}

\bibitem[\citeproctext]{ref-beckermanForagingBiologyPredicts2006}
Beckerman, A. P., Petchey, O. L., \& Warren, P. H. (2006). Foraging
biology predicts food web complexity. \emph{Proceedings of the National
Academy of Sciences}, \emph{103}(37), 13745--13749.
\url{https://doi.org/10.1073/pnas.0603039103}

\bibitem[\citeproctext]{ref-berlowGoldilocksFactorFood2008}
Berlow, E. L., Brose, U., \& Martinez, N. D. (2008). The {``{Goldilocks}
factor''} in food webs. \emph{Proceedings of the National Academy of
Sciences}, \emph{105}(11), 4079--4080.
\url{https://doi.org/10.1073/pnas.0800967105}

\bibitem[\citeproctext]{ref-berlowInteractionStrengthsFood2004}
Berlow, E. L., Neutel, A.-M., Cohen, J. E., de Ruiter, P. C., Ebenman,
B., Emmerson, M., Fox, J. W., Jansen, V. A. A., Iwan Jones, J.,
Kokkoris, G. D., Logofet, D. O., McKane, A. J., Montoya, J. M., \&
Petchey, O. (2004). Interaction strengths in food webs: Issues and
opportunities. \emph{Journal of Animal Ecology}, \emph{73}(3), 585--598.
\url{https://doi.org/10.1111/j.0021-8790.2004.00833.x}

\bibitem[\citeproctext]{ref-bitonInductiveLinkPrediction2024}
Biton, B., Puzis, R., \& Pilosof, S. (2024). \emph{Inductive link
prediction boosts data availability and enables cross-community link
prediction in ecological networks}.

\bibitem[\citeproctext]{ref-blanchetCooccurrenceNotEvidence2020}
Blanchet, F. G., Cazelles, K., \& Gravel, D. (2020). Co-occurrence is
not evidence of ecological interactions. \emph{Ecology Letters},
\emph{23}(7), 1050--1063. \url{https://doi.org/10.1111/ele.13525}

\bibitem[\citeproctext]{ref-brimacombeApplyingMethodIts2024}
Brimacombe, C., Bodner, K., \& Fortin, M.-J. (2024). \emph{Applying a
method before its proof-of-concept: {A} cautionary tale using inferred
food webs}. \url{https://doi.org/10.13140/RG.2.2.22076.65927}

\bibitem[\citeproctext]{ref-brimacombeShortcomingsReusingSpecies2023}
Brimacombe, C., Bodner, K., Michalska-Smith, M., Poisot, T., \& Fortin,
M.-J. (2023). Shortcomings of reusing species interaction networks
created by different sets of researchers. \emph{PLOS Biology},
\emph{21}(4), e3002068.
\url{https://doi.org/10.1371/journal.pbio.3002068}

\bibitem[\citeproctext]{ref-canardEmergenceStructuralPatterns2012}
Canard, E., Mouquet, N., Marescot, L., Gaston, K. J., Gravel, D., \&
Mouillot, D. (2012). Emergence of {Structural Patterns} in {Neutral
Trophic Networks}. \emph{PLOS ONE}, \emph{7}(8), e38295.
\url{https://doi.org/10.1371/journal.pone.0038295}

\bibitem[\citeproctext]{ref-caronTraitmatchingModelsPredict2024}
Caron, D., Brose, U., Lurgi, M., Blanchet, F. G., Gravel, D., \&
Pollock, L. J. (2024). Trait-matching models predict pairwise
interactions across regions, not food web properties. \emph{Global
Ecology and Biogeography}, \emph{33}(4), e13807.
\url{https://doi.org/10.1111/geb.13807}

\bibitem[\citeproctext]{ref-cherifEnvironmentRescueCan2024}
Cherif, M., Brose, U., Hirt, M. R., Ryser, R., Silve, V., Albert, G.,
Arnott, R., Berti, E., Cirtwill, A., Dyer, A., Gauzens, B., Gupta, A.,
Ho, H.-C., Portalier, S. M. J., Wain, D., \& Wootton, K. (2024). The
environment to the rescue: Can physics help predict predator--prey
interactions? \emph{Biological Reviews}, \emph{n/a}(n/a).
\url{https://doi.org/10.1111/brv.13105}

\bibitem[\citeproctext]{ref-cleggImpactIntraspecificVariation2018}
Clegg, T., Ali, M., \& Beckerman, A. P. (2018). The impact of
intraspecific variation on food web structure. \emph{Ecology},
\emph{99}(12), 2712--2720. \url{https://doi.org/10.1002/ecy.2523}

\bibitem[\citeproctext]{ref-curtsdotterEcosystemFunctionPredator2019}
Curtsdotter, A., Banks, H. T., Banks, J. E., Jonsson, M., Jonsson, T.,
Laubmeier, A. N., Traugott, M., \& Bommarco, R. (2019). Ecosystem
function in predator--prey food webs---confronting dynamic models with
empirical data. \emph{Journal of Animal Ecology}, \emph{88}(2),
196--210. \url{https://doi.org/10.1111/1365-2656.12892}

\bibitem[\citeproctext]{ref-dallarivaExploringEvolutionarySignature2016}
Dalla Riva, G. V., \& Stouffer, D. B. (2016). Exploring the evolutionary
signature of food webs' backbones using functional traits. \emph{Oikos},
\emph{125}(4), 446--456. \url{https://doi.org/10.1111/oik.02305}

\bibitem[\citeproctext]{ref-delmasAnalysingEcologicalNetworks2019}
Delmas, E., Besson, M., Brice, M.-H., Burkle, L. A., Riva, G. V. D.,
Fortin, M.-J., Gravel, D., Guimarães, P. R., Hembry, D. H., Newman, E.
A., Olesen, J. M., Pires, M. M., Yeakel, J. D., \& Poisot, T. (2019).
Analysing ecological networks of species interactions. \emph{Biological
Reviews}, \emph{94}(1), 16--36. \url{https://doi.org/10.1111/brv.12433}

\bibitem[\citeproctext]{ref-delmasSimulationsBiomassDynamics2017}
Delmas, E., Brose, U., Gravel, D., Stouffer, D. B., \& Poisot, T.
(2017). Simulations of biomass dynamics in community food webs.
\emph{Methods in Ecology and Evolution}, \emph{8}(7), 881--886.
\url{https://doi.org/10.1111/2041-210X.12713}

\bibitem[\citeproctext]{ref-dunnSixthMassCoextinction2009}
Dunn, R. R., Harris, N. C., Colwell, R. K., Koh, L. P., \& Sodhi, N. S.
(2009). The sixth mass coextinction: Are most endangered species
parasites and mutualists? \emph{Proceedings. Biological Sciences},
\emph{276}(1670), 3037--3045.
\url{https://doi.org/10.1098/rspb.2009.0413}

\bibitem[\citeproctext]{ref-dunneNetworkStructureFood2006}
Dunne, J. A. (2006). The {Network Structure} of {Food Webs}. In J. A.
Dunne \& M. Pascual (Eds.), \emph{Ecological networks: {Linking}
structure and dynamics} (pp. 27--86). Oxford University Press.

\bibitem[\citeproctext]{ref-estayEditorialPatternsProcesses2023}
Estay, S. A., Fortin, M.-J., \& López, D. N. (2023). Editorial:
{Patterns} and processes in ecological networks over space.
\emph{Frontiers in Ecology and Evolution}, \emph{11}.

\bibitem[\citeproctext]{ref-frickeCollapseTerrestrialMammal2022}
Fricke, E. C., Hsieh, C., Middleton, O., Gorczynski, D., Cappello, C.
D., Sanisidro, O., Rowan, J., Svenning, J.-C., \& Beaudrot, L. (2022).
Collapse of terrestrial mammal food webs since the {Late Pleistocene}.
\emph{Science}, \emph{377}(6609), 1008--1011.
\url{https://doi.org/10.1126/science.abn4012}

\bibitem[\citeproctext]{ref-golubskiModifyingModifiersWhat2011}
Golubski, A. J., \& Abrams, P. A. (2011). Modifying modifiers: What
happens when interspecific interactions interact? \emph{Journal of
Animal Ecology}, \emph{80}(5), 1097--1108.
\url{https://doi.org/10.1111/j.1365-2656.2011.01852.x}

\bibitem[\citeproctext]{ref-gomezEcologicalInteractionsAre2010}
Gómez, J. M., Verdú, M., \& Perfectti, F. (2010). Ecological
interactions are evolutionarily conserved across the entire tree of
life. \emph{Nature}, \emph{465}(7300), 918--921.
\url{https://doi.org/10.1038/nature09113}

\bibitem[\citeproctext]{ref-higinoMismatchIUCNRange2023}
Higino, G. T., Banville, F., Dansereau, G., Muñoz, N. R. F., Windsor,
F., \& Poisot, T. (2023). Mismatch between {IUCN} range maps and species
interactions data illustrated using the {Serengeti} food web.
\emph{PeerJ}, \emph{11}, e14620.
\url{https://doi.org/10.7717/peerj.14620}

\bibitem[\citeproctext]{ref-hutchinsonSeeingForestTrees2019}
Hutchinson, M. C., Bramon Mora, B., Pilosof, S., Barner, A. K., Kéfi,
S., Thébault, E., Jordano, P., \& Stouffer, D. B. (2019). Seeing the
forest for the trees: {Putting} multilayer networks to work for
community ecology. \emph{Functional Ecology}, \emph{33}(2), 206--217.
\url{https://doi.org/10.1111/1365-2435.13237}

\bibitem[\citeproctext]{ref-jordanoChasingEcologicalInteractions2016}
Jordano, P. (2016a). Chasing {Ecological Interactions}. \emph{PLOS
Biology}, \emph{14}(9), e1002559.
\url{https://doi.org/10.1371/journal.pbio.1002559}

\bibitem[\citeproctext]{ref-jordanoSamplingNetworksEcological2016}
Jordano, P. (2016b). Sampling networks of ecological interactions.
\emph{Functional Ecology}. \url{https://doi.org/10.1111/1365-2435.12763}

\bibitem[\citeproctext]{ref-kefiNetworkStructureFood2015}
Kéfi, S., Berlow, E. L., Wieters, E. A., Joppa, L. N., Wood, S. A.,
Brose, U., \& Navarrete, S. A. (2015). Network structure beyond food
webs: Mapping non-trophic and trophic interactions on {Chilean} rocky
shores. \emph{Ecology}, \emph{96}(1), 291--303.
\url{https://doi.org/10.1890/13-1424.1}

\bibitem[\citeproctext]{ref-kefiMoreMealIntegrating2012}
Kéfi, S., Berlow, E. L., Wieters, E. A., Navarrete, S. A., Petchey, O.
L., Wood, S. A., Boit, A., Joppa, L. N., Lafferty, K. D., Williams, R.
J., Martinez, N. D., Menge, B. A., Blanchette, C. A., Iles, A. C., \&
Brose, U. (2012). More than a meal{\ldots{}} integrating non-feeding
interactions into food webs: {More} than a meal {\ldots{}}.
\emph{Ecology Letters}, \emph{15}(4), 291--300.
\url{https://doi.org/10.1111/j.1461-0248.2011.01732.x}

\bibitem[\citeproctext]{ref-krauseCompartmentsRevealedFoodweb2003}
Krause, A. E., Frank, K. A., Mason, D. M., Ulanowicz, R. E., \& Taylor,
W. W. (2003). Compartments revealed in food-web structure.
\emph{Nature}, \emph{426}(6964), 282--285.
\url{https://doi.org/10.1038/nature02115}

\bibitem[\citeproctext]{ref-lajaaitiEcologicalNetworksDynamicsJlJulia2024}
Lajaaiti, I., Bonnici, I., Kéfi, S., Mayall, H., Danet, A., Beckerman,
A. P., Malpas, T., \& Delmas, E. (2024).
\emph{{EcologicalNetworksDynamics}.jl {A Julia} package to simulate the
temporal dynamics of complex ecological networks} (p.
2024.03.20.585899). bioRxiv.
\url{https://doi.org/10.1101/2024.03.20.585899}

\bibitem[\citeproctext]{ref-lindemanTrophicDynamicAspectEcology1942}
Lindeman, R. L. (1942). The {Trophic-Dynamic Aspect} of {Ecology}.
\emph{Ecology}, \emph{23}(4), 399--417.
\url{https://doi.org/10.2307/1930126}

\bibitem[\citeproctext]{ref-maioranoTETRAEUSpecieslevelTrophic2020}
Maiorano, L., Montemaggiori, A., Ficetola, G. F., O'Connor, L., \&
Thuiller, W. (2020). {TETRA-EU} 1.0: {A} species-level trophic metaweb
of {European} tetrapods. \emph{Global Ecology and Biogeography},
\emph{29}(9), 1452--1457. \url{https://doi.org/10.1111/geb.13138}

\bibitem[\citeproctext]{ref-momalTreebasedInferenceSpecies2020}
Momal, R., Robin, S., \& Ambroise, C. (2020). Tree-based inference of
species interaction networks from abundance data. \emph{Methods in
Ecology and Evolution}, \emph{11}(5), 621--632.
\url{https://doi.org/10.1111/2041-210X.13380}

\bibitem[\citeproctext]{ref-pawarDimensionalityConsumerSearch2012}
Pawar, S., Dell, A. I., \& Savage, V. M. (2012). Dimensionality of
consumer search space drives trophic interaction strengths.
\emph{Nature}, \emph{486}(7404), 485--489.
\url{https://doi.org/10.1038/nature11131}

\bibitem[\citeproctext]{ref-petcheySizeForagingFood2008}
Petchey, O. L., Beckerman, A. P., Riede, J. O., \& Warren, P. H. (2008).
Size, foraging, and food web structure. \emph{Proceedings of the
National Academy of Sciences}, \emph{105}(11), 4191--4196.
\url{https://doi.org/10.1073/pnas.0710672105}

\bibitem[\citeproctext]{ref-petcheyFitEfficiencyBiology2011}
Petchey, O. L., Beckerman, A. P., Riede, J. O., \& Warren, P. H. (2011).
Fit, efficiency, and biology: {Some} thoughts on judging food web
models. \emph{Journal of Theoretical Biology}, \emph{279}(1), 169--171.
\url{https://doi.org/10.1016/j.jtbi.2011.03.019}

\bibitem[\citeproctext]{ref-pilosofMultilayerNatureEcological2017}
Pilosof, S., Porter, M. A., Pascual, M., \& Kéfi, S. (2017). The
multilayer nature of ecological networks. \emph{Nature Ecology \&
Evolution}, \emph{1}(4), 101.
\url{https://doi.org/10.1038/s41559-017-0101}

\bibitem[\citeproctext]{ref-poisotGuidelinesPredictionSpecies2023}
Poisot, T. (2023). Guidelines for the prediction of species interactions
through binary classification. \emph{Methods in Ecology and Evolution},
\emph{14}(5), 1333--1345. \url{https://doi.org/10.1111/2041-210X.14071}

\bibitem[\citeproctext]{ref-poisotGlobalKnowledgeGaps2021}
Poisot, T., Bergeron, G., Cazelles, K., Dallas, T., Gravel, D.,
MacDonald, A., Mercier, B., Violet, C., \& Vissault, S. (2021). Global
knowledge gaps in species interaction networks data. \emph{Journal of
Biogeography}, \emph{48}(7), 1552--1563.
\url{https://doi.org/10.1111/jbi.14127}

\bibitem[\citeproctext]{ref-poisotStructureProbabilisticNetworks2016}
Poisot, T., Cirtwill, A., Cazelles, K., Gravel, D., Fortin, M.-J., \&
Stouffer, D. (2016). The structure of probabilistic networks.
\emph{Methods in Ecology and Evolution}, \emph{7}(3), 303--312.
\url{https://doi.org/10}

\bibitem[\citeproctext]{ref-poisotSpeciesWhyEcological2015}
Poisot, T., Stouffer, D. B., \& Gravel, D. (2015). Beyond species: Why
ecological interaction networks vary through space and time.
\emph{Oikos}, \emph{124}(3), 243--251.
\url{https://doi.org/10.1111/oik.01719}

\bibitem[\citeproctext]{ref-poisotDescribeUnderstandPredict2016}
Poisot, T., Stouffer, D. B., \& Kéfi, S. (2016). Describe, understand
and predict: Why do we need networks in ecology? \emph{Functional
Ecology}, \emph{30}(12), 1878--1882.
\url{https://www.jstor.org/stable/48582345}

\bibitem[\citeproctext]{ref-pollockUnderstandingCooccurrenceModelling2014}
Pollock, L. J., Tingley, R., Morris, W. K., Golding, N., O'Hara, R. B.,
Parris, K. M., Vesk, P. A., \& McCarthy, M. A. (2014). Understanding
co-occurrence by modelling species simultaneously with a {Joint Species
Distribution Model} ({JSDM}). \emph{Methods in Ecology and Evolution},
\emph{5}(5), 397--406. \url{https://doi.org/10.1111/2041-210X.12180}

\bibitem[\citeproctext]{ref-pringleUntanglingFoodWebs2020}
Pringle, R. M. (2020). Untangling {Food Webs}. In \emph{Unsolved
{Problems} in {Ecology}} (pp. 225--238). Princeton University Press.
\url{https://doi.org/10.1515/9780691195322-020}

\bibitem[\citeproctext]{ref-pringleResolvingFoodWebStructure2020}
Pringle, R. M., \& Hutchinson, M. C. (2020). Resolving {Food-Web
Structure}. \emph{Annual Review of Ecology, Evolution and Systematics},
\emph{51}(Volume 51, 2020), 55--80.
\url{https://doi.org/10.1146/annurev-ecolsys-110218-024908}

\bibitem[\citeproctext]{ref-proulxNetworkThinkingEcology2005}
Proulx, S. R., Promislow, D. E. L., \& Phillips, P. C. (2005). Network
thinking in ecology and evolution. \emph{Trends in Ecology \&
Evolution}, \emph{20}(6), 345--353.
\url{https://doi.org/10.1016/j.tree.2005.04.004}

\bibitem[\citeproctext]{ref-rooneyLandscapeTheoryFood2008}
Rooney, N., McCann, K. S., \& Moore, J. C. (2008). A landscape theory
for food web architecture. \emph{Ecology Letters}, \emph{11}(8),
867--881. \url{https://doi.org/10.1111/j.1461-0248.2008.01193.x}

\bibitem[\citeproctext]{ref-roopnarineExtinctionCascadesCatastrophe2006}
Roopnarine, P. D. (2006). Extinction {Cascades} and {Catastrophe} in
{Ancient Food Webs}. \emph{Paleobiology}, \emph{32}(1), 1--19.
\url{https://www.jstor.org/stable/4096814}

\bibitem[\citeproctext]{ref-saraviaEcologicalNetworkAssembly2022}
Saravia, L. A., Marina, T. I., Kristensen, N. P., De Troch, M., \& Momo,
F. R. (2022). Ecological network assembly: {How} the regional metaweb
influences local food webs. \emph{Journal of Animal Ecology},
\emph{91}(3), 630--642. \url{https://doi.org/10.1111/1365-2656.13652}

\bibitem[\citeproctext]{ref-segarRoleEvolutionShaping2020}
Segar, S. T., Fayle, T. M., Srivastava, D. S., Lewinsohn, T. M., Lewis,
O. T., Novotny, V., Kitching, R. L., \& Maunsell, S. C. (2020). The
{Role} of {Evolution} in {Shaping Ecological Networks}. \emph{Trends in
Ecology \& Evolution}, \emph{35}(5), 454--466.
\url{https://doi.org/10.1016/j.tree.2020.01.004}

\bibitem[\citeproctext]{ref-songRigorousValidationEcological2024}
Song, C., \& Levine, J. M. (2024). \emph{Rigorous (in)validation of
ecological models} (p. 2024.09.19.613075). bioRxiv.
\url{https://doi.org/10.1101/2024.09.19.613075}

\bibitem[\citeproctext]{ref-stockPairwiseLearningPredicting2021}
Stock, M. (2021). Pairwise learning for predicting pollination
interactions based on traits and phylogeny. \emph{Ecological Modelling},
14.

\bibitem[\citeproctext]{ref-stoufferAllEcologicalModels2019}
Stouffer, D. B. (2019). All ecological models are wrong, but some are
useful. \emph{Journal of Animal Ecology}, \emph{88}(2), 192--195.
\url{https://doi.org/10.1111/1365-2656.12949}

\bibitem[\citeproctext]{ref-strydomFoodWebReconstruction2022}
Strydom, T., Bouskila, S., Banville, F., Barros, C., Caron, D., Farrell,
M. J., Fortin, M.-J., Hemming, V., Mercier, B., Pollock, L. J., Runghen,
R., Dalla Riva, G. V., \& Poisot, T. (2022). Food web reconstruction
through phylogenetic transfer of low-rank network representation.
\emph{Methods in Ecology and Evolution}, \emph{13}(12), 2838--2849.
\url{https://doi.org/10.1111/2041-210X.13835}

\bibitem[\citeproctext]{ref-strydomGraphEmbeddingTransfer2023}
Strydom, T., Bouskila, S., Banville, F., Barros, C., Caron, D., Farrell,
M. J., Fortin, M.-J., Mercier, B., Pollock, L. J., Runghen, R., Dalla
Riva, G. V., \& Poisot, T. (2023). Graph embedding and transfer learning
can help predict potential species interaction networks despite data
limitations. \emph{Methods in Ecology and Evolution}, \emph{14}(12),
2917--2930. \url{https://doi.org/10.1111/2041-210X.14228}

\bibitem[\citeproctext]{ref-strydomRoadmapPredictingSpecies2021}
Strydom, T., Catchen, M. D., Banville, F., Caron, D., Dansereau, G.,
Desjardins-Proulx, P., Forero-Muñoz, N. R., Higino, G., Mercier, B.,
Gonzalez, A., Gravel, D., Pollock, L., \& Poisot, T. (2021). A roadmap
towards predicting species interaction networks (across space and time).
\emph{Philosophical Transactions of the Royal Society B: Biological
Sciences}, \emph{376}(1837), 20210063.
\url{https://doi.org/10.1098/rstb.2021.0063}

\bibitem[\citeproctext]{ref-terryFindingMissingLinks2020}
Terry, J. C. D., \& Lewis, O. T. (2020). Finding missing links in
interaction networks. \emph{Ecology}, \emph{101}(7), e03047.
\url{https://doi.org/10.1002/ecy.3047}

\bibitem[\citeproctext]{ref-williamsSimpleRulesYield2000}
Williams, R. J., \& Martinez, N. D. (2000). Simple rules yield complex
food webs. \emph{Nature}, \emph{404}(6774), 180--183.
\url{https://doi.org/10.1038/35004572}

\bibitem[\citeproctext]{ref-williamsSuccessItsLimits2008}
Williams, R. J., \& Martinez, N. D. (2008). Success and its limits among
structural models of complex food webs. \emph{Journal of Animal
Ecology}, \emph{77}(3), 512--519.
\url{https://doi.org/10.1111/j.1365-2656.2008.01362.x}

\bibitem[\citeproctext]{ref-woottonModularTheoryTrophic2023}
Wootton, K. L., Curtsdotter, A., Roslin, T., Bommarco, R., \& Jonsson,
T. (2023). Towards a modular theory of trophic interactions.
\emph{Functional Ecology}, \emph{37}(1), 26--43.
\url{https://doi.org/10.1111/1365-2435.13954}

\bibitem[\citeproctext]{ref-xieCompletenessCommunityStructure2017}
Xie, J.-R., Zhang, P., Zhang, H.-F., \& Wang, B.-H. (2017). Completeness
of {Community Structure} in {Networks}. \emph{Scientific Reports},
\emph{7}(1), 5269. \url{https://doi.org/10.1038/s41598-017-05585-6}

\bibitem[\citeproctext]{ref-yodzisCompartmentationRealAssembled1982}
Yodzis, P. (1982). The {Compartmentation} of {Real} and {Assembled
Ecosystems}. \emph{The American Naturalist}, \emph{120}(5), 551--570.
\url{https://doi.org/10.1086/284013}

\end{CSLReferences}




\end{document}
