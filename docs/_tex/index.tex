% Options for packages loaded elsewhere
\PassOptionsToPackage{unicode}{hyperref}
\PassOptionsToPackage{hyphens}{url}
\PassOptionsToPackage{dvipsnames,svgnames,x11names}{xcolor}
%
\documentclass[
]{article}

\usepackage{amsmath,amssymb}
\usepackage{iftex}
\ifPDFTeX
  \usepackage[T1]{fontenc}
  \usepackage[utf8]{inputenc}
  \usepackage{textcomp} % provide euro and other symbols
\else % if luatex or xetex
  \usepackage{unicode-math}
  \defaultfontfeatures{Scale=MatchLowercase}
  \defaultfontfeatures[\rmfamily]{Ligatures=TeX,Scale=1}
\fi
\usepackage{lmodern}
\ifPDFTeX\else  
    % xetex/luatex font selection
\fi
% Use upquote if available, for straight quotes in verbatim environments
\IfFileExists{upquote.sty}{\usepackage{upquote}}{}
\IfFileExists{microtype.sty}{% use microtype if available
  \usepackage[]{microtype}
  \UseMicrotypeSet[protrusion]{basicmath} % disable protrusion for tt fonts
}{}
\makeatletter
\@ifundefined{KOMAClassName}{% if non-KOMA class
  \IfFileExists{parskip.sty}{%
    \usepackage{parskip}
  }{% else
    \setlength{\parindent}{0pt}
    \setlength{\parskip}{6pt plus 2pt minus 1pt}}
}{% if KOMA class
  \KOMAoptions{parskip=half}}
\makeatother
\usepackage{xcolor}
\setlength{\emergencystretch}{3em} % prevent overfull lines
\setcounter{secnumdepth}{5}
% Make \paragraph and \subparagraph free-standing
\ifx\paragraph\undefined\else
  \let\oldparagraph\paragraph
  \renewcommand{\paragraph}[1]{\oldparagraph{#1}\mbox{}}
\fi
\ifx\subparagraph\undefined\else
  \let\oldsubparagraph\subparagraph
  \renewcommand{\subparagraph}[1]{\oldsubparagraph{#1}\mbox{}}
\fi


\providecommand{\tightlist}{%
  \setlength{\itemsep}{0pt}\setlength{\parskip}{0pt}}\usepackage{longtable,booktabs,array}
\usepackage{calc} % for calculating minipage widths
% Correct order of tables after \paragraph or \subparagraph
\usepackage{etoolbox}
\makeatletter
\patchcmd\longtable{\par}{\if@noskipsec\mbox{}\fi\par}{}{}
\makeatother
% Allow footnotes in longtable head/foot
\IfFileExists{footnotehyper.sty}{\usepackage{footnotehyper}}{\usepackage{footnote}}
\makesavenoteenv{longtable}
\usepackage{graphicx}
\makeatletter
\def\maxwidth{\ifdim\Gin@nat@width>\linewidth\linewidth\else\Gin@nat@width\fi}
\def\maxheight{\ifdim\Gin@nat@height>\textheight\textheight\else\Gin@nat@height\fi}
\makeatother
% Scale images if necessary, so that they will not overflow the page
% margins by default, and it is still possible to overwrite the defaults
% using explicit options in \includegraphics[width, height, ...]{}
\setkeys{Gin}{width=\maxwidth,height=\maxheight,keepaspectratio}
% Set default figure placement to htbp
\makeatletter
\def\fps@figure{htbp}
\makeatother
% definitions for citeproc citations
\NewDocumentCommand\citeproctext{}{}
\NewDocumentCommand\citeproc{mm}{%
  \begingroup\def\citeproctext{#2}\cite{#1}\endgroup}
\makeatletter
 % allow citations to break across lines
 \let\@cite@ofmt\@firstofone
 % avoid brackets around text for \cite:
 \def\@biblabel#1{}
 \def\@cite#1#2{{#1\if@tempswa , #2\fi}}
\makeatother
\newlength{\cslhangindent}
\setlength{\cslhangindent}{1.5em}
\newlength{\csllabelwidth}
\setlength{\csllabelwidth}{3em}
\newenvironment{CSLReferences}[2] % #1 hanging-indent, #2 entry-spacing
 {\begin{list}{}{%
  \setlength{\itemindent}{0pt}
  \setlength{\leftmargin}{0pt}
  \setlength{\parsep}{0pt}
  % turn on hanging indent if param 1 is 1
  \ifodd #1
   \setlength{\leftmargin}{\cslhangindent}
   \setlength{\itemindent}{-1\cslhangindent}
  \fi
  % set entry spacing
  \setlength{\itemsep}{#2\baselineskip}}}
 {\end{list}}
\usepackage{calc}
\newcommand{\CSLBlock}[1]{\hfill\break\parbox[t]{\linewidth}{\strut\ignorespaces#1\strut}}
\newcommand{\CSLLeftMargin}[1]{\parbox[t]{\csllabelwidth}{\strut#1\strut}}
\newcommand{\CSLRightInline}[1]{\parbox[t]{\linewidth - \csllabelwidth}{\strut#1\strut}}
\newcommand{\CSLIndent}[1]{\hspace{\cslhangindent}#1}

\makeatletter
\@ifpackageloaded{tcolorbox}{}{\usepackage[skins,breakable]{tcolorbox}}
\@ifpackageloaded{fontawesome5}{}{\usepackage{fontawesome5}}
\definecolor{quarto-callout-color}{HTML}{909090}
\definecolor{quarto-callout-note-color}{HTML}{0758E5}
\definecolor{quarto-callout-important-color}{HTML}{CC1914}
\definecolor{quarto-callout-warning-color}{HTML}{EB9113}
\definecolor{quarto-callout-tip-color}{HTML}{00A047}
\definecolor{quarto-callout-caution-color}{HTML}{FC5300}
\definecolor{quarto-callout-color-frame}{HTML}{acacac}
\definecolor{quarto-callout-note-color-frame}{HTML}{4582ec}
\definecolor{quarto-callout-important-color-frame}{HTML}{d9534f}
\definecolor{quarto-callout-warning-color-frame}{HTML}{f0ad4e}
\definecolor{quarto-callout-tip-color-frame}{HTML}{02b875}
\definecolor{quarto-callout-caution-color-frame}{HTML}{fd7e14}
\makeatother
\makeatletter
\@ifpackageloaded{caption}{}{\usepackage{caption}}
\AtBeginDocument{%
\ifdefined\contentsname
  \renewcommand*\contentsname{Table of contents}
\else
  \newcommand\contentsname{Table of contents}
\fi
\ifdefined\listfigurename
  \renewcommand*\listfigurename{List of Figures}
\else
  \newcommand\listfigurename{List of Figures}
\fi
\ifdefined\listtablename
  \renewcommand*\listtablename{List of Tables}
\else
  \newcommand\listtablename{List of Tables}
\fi
\ifdefined\figurename
  \renewcommand*\figurename{Figure}
\else
  \newcommand\figurename{Figure}
\fi
\ifdefined\tablename
  \renewcommand*\tablename{Table}
\else
  \newcommand\tablename{Table}
\fi
}
\@ifpackageloaded{float}{}{\usepackage{float}}
\floatstyle{ruled}
\@ifundefined{c@chapter}{\newfloat{codelisting}{h}{lop}}{\newfloat{codelisting}{h}{lop}[chapter]}
\floatname{codelisting}{Listing}
\newcommand*\listoflistings{\listof{codelisting}{List of Listings}}
\makeatother
\makeatletter
\makeatother
\makeatletter
\@ifpackageloaded{caption}{}{\usepackage{caption}}
\@ifpackageloaded{subcaption}{}{\usepackage{subcaption}}
\makeatother
\ifLuaTeX
  \usepackage{selnolig}  % disable illegal ligatures
\fi
\usepackage{bookmark}

\IfFileExists{xurl.sty}{\usepackage{xurl}}{} % add URL line breaks if available
\urlstyle{same} % disable monospaced font for URLs
\hypersetup{
  pdftitle={Unveiling the Complexity of Food Webs: A Comprehensive Overview of Definitions, Scales, and Mechanisms},
  pdfauthor={Tanya Strydom; Jennifer A. Dunne; Timothée Poisot; Andrew P. Beckerman},
  pdfkeywords={food web, network construction, scientific ignorance},
  colorlinks=true,
  linkcolor={blue},
  filecolor={Maroon},
  citecolor={Blue},
  urlcolor={Blue},
  pdfcreator={LaTeX via pandoc}}


\title{Unveiling the Complexity of Food Webs: A Comprehensive Overview
of Definitions, Scales, and Mechanisms}
\author{Tanya Strydom %
%
\textsuperscript{%
%
1%
}%
; Jennifer A. Dunne %
%
\textsuperscript{%
%
2%
}%
; Timothée Poisot %
%
\textsuperscript{%
3,%
4%
}%
; Andrew P. Beckerman %
%
\textsuperscript{%
%
1%
}%
}
\date{2024-10-05}

\usepackage{setspace}
\usepackage[left]{lineno}
\usepackage[letterpaper]{geometry}

\usepackage[nolists,noheads,markers]{endfloat}
\geometry{margin=2.5cm}

\begin{document}

\thispagestyle{empty}
{\bfseries\sffamily\Large Unveiling the Complexity of Food Webs: A
Comprehensive Overview of Definitions, Scales, and Mechanisms}
\vfil
Tanya Strydom %
%
\textsuperscript{%
%
1%
}%
; Jennifer A. Dunne %
%
\textsuperscript{%
%
2%
}%
; Timothée Poisot %
%
\textsuperscript{%
3,%
4%
}%
; Andrew P. Beckerman %
%
\textsuperscript{%
%
1%
}%

\vfil
{\small
\textbf{Abstract:} Food webs are a useful abstraction and representation
of the feeding links between species in a community and are used to
infer many ecosystem level processes. However, the different theories,
mechanisms, and criteria that underpin how a food web is defined and,
ultimately, constructed means that not all food webs are representing
the same ecological process. Here we present a synthesis of the
different assumptions, scales and mechanisms that are used to define
different ecological networks ranging from metawebs (an inventory of all
potential interactions) to fully realised networks (interactions that
occur within a given community over a certain timescale). Illuminating
the assumptions, scales, and mechanisms of network inference allows a
formal categorisation of how to use networks to answer key ecological
and conservation questions and defines guidelines to prevent
unintentional misuse or misinterpretation.
\vfil
\textbf{Keywords:} %
food web, network construction, %
scientific ignorance%
}
\clearpage
\setcounter{page}{1}
\doublespacing
\linenumbers

At the heart of modern biodiversity science are a set of concepts and
theories about biodiversity, stability and function. These relate to the
abundance, distribution and services that biodiversity provides, and how
biodiversity -- as an interconnected set of species -- responds to
multiple stressors. The interaction between species is one of the
fundamental building blocks of ecological communities, providing a
powerful abstraction that can help quantify, conceptualise, and
understand biodiversity dynamics, and ultimately, make predictions,
mitigate change, and manage services. Such network representations of
biodiversity (including within species diversity) are increasingly
argued to be an asset to predictive ecology, climate change mitigation
and resource management. Here, it is argued that characterising
biodiversity in a network will allow deeper capacity to understand and
predict the abundance, distribution, dynamics and services provided by
multiple species facing multiple stressors.

A `network' can be defined and conceptualised in a myriad of ways, which
means that different networks will be embedding different processes (or
determinants) of interactions, ultimately influencing the patterns and
mechanisms that are inferred (Brimacombe et al., 2023; Proulx et al.,
2005). The different ways in which a network can be represented is the
result of \emph{how} the network is constructed, which itself rests on
two pillars: the data used to construct the network and the underlying
theory as to what drives the interactions between species. The latter
represents an expression of mechanism and process that gives rise to the
patterns that emerge from collating interactions among species, and will
ultimately inform which data are deemed important in the determination
of interactions occurring. Each of these pillars carries with it a set
of practical, semantic and conceptual constraints that not only
influence progress in making network ecology more valuable and
potentially predictive, but help define the spatial, temporal, and
evolutionary scale of assumptions we make and the predictions we might
generate from different network representations.

In this perspective we aim to provide an overview of the different
\textbf{food web} representations, particularly how these relate to the
terminology used to define a network, and how this influenced by both
the processes that determine networks as well as how this relates to the
way in which we construct networks. The provision of this detail
ultimately leads to a set of insights and conclusions about whether,
when and under what conditions network representations of biodiversity
can contribute to the advancement of ecological theory and generate
value in predictive ecology. Specifically, we finish this perspective
with an overview of fundamental questions in ecology that we think can
benefit from network thinking and a proposal that such thinking can
accelerate our capacity to predict the impact of multiple stressors on
biodiverse communities.

\section{Setting the Scene: The Not So Basics of Nodes and
Edges}\label{sec-anatomy}

Defining a food web seems simple; it is the representation of the
interactions (edges) between species (nodes), however the definition of
`edges' and `nodes', as well as the scale at which they are aggregated
can take many forms (Poisot, Stouffer, et al., 2016), which ultimately
encodes a series of assumptions and criteria within a network. An
awareness of variance in the way a food web can be defined is critical
as a network (or its adjacency matrix) is both the `object' from which
inferences are made (\emph{e.g.,} the interactions between species, or
how the structure influences ecosystem level processes) as well as the
`product' of either the data collection (Brimacombe et al., 2023) or
prediction process (Banville et al., 2024). One thus needs to be aware
of both the criteria that is used to define nodes and edges, and what
processes or mechanisms the aggregation of the two represents, as this
will determine what the network can be used for.

\subsubsection{How do we define a node?}\label{how-do-we-define-a-node}

Although this may seem an elementary question in the context of food
webs --- a node \emph{should} represent a (taxonomic) species, the
reality is that nodes can often represent an aggregation of different
species - so called `trophic species' (Williams \& Martinez, 2000;
Yodzis, 1982) or segregation of species by life stages (Clegg et al.,
2018). Practical implications of how we are aggregating the nodes is
that the resolution may not always be `pixel perfect', which limits the
ability to make (taxonomic) species specific inferences \emph{e.g.,}
does species \(a\) eat species \(b\), however there is value in having
nodes that represent an aggregation of species, as the distribution of
the links between them are more meaningful in terms of understanding
energy flow and distribution within the system.

\subsubsection{What is captured by an
edge?}\label{what-is-captured-by-an-edge}

At its core, links within food webs can be thought of as a
representation of either feeding links between species - be that
realised (Pringle, 2020) or potential (Dunne, 2006), alternative links
can represent fluxes within the system \emph{e.g.,} energy transfer or
material flow as the result of the feeding links between species
(Lindeman, 1942). Fundamentally this means that the links within a
network represent different `currencies' (either the feasibility of a
link existing between two species or the energy that is moving through
the system) and how the links within a network are specified will
influence the resulting structure of the network. For example taking a
food web that consists of links representing all \emph{potential}
feeding links for a community (\emph{i.e.,} a metaweb) will be
meaningless if one is interested in understanding the flow of energy
through the network as the links within a metaweb do not represent
environmental/energetic constraints, making them poor representations of
which interactions are \emph{realised} in a specific location (Caron et
al., 2024). In addition to the various ways of defining the links
between species pairs there are also a myriad of ways in which the links
themselves can be quantified. Links between species are often treated as
being present or absent (\emph{i.e.,} binary) but it is also possible to
use probabilities (Banville et al., 2024; which quantifies how likely an
interaction is to occur, Poisot, Cirtwill, et al., 2016) or continuous
measurements (which quantifies the strength of of an interaction, Berlow
et al., 2004).

\subsubsection{Network representations}\label{network-representations}

Broadly, networks can be thought of to fall into two different `types';
namely metawebs; traditionally defined as all of the \emph{potential}
interactions for a specific species pool (Dunne, 2006), and realised
networks; which is the subset of interactions in a metaweb that are
\emph{realised} at a given time and place. The fundamental difference
between these two different network representations is that a metaweb
provides insight as to the viability of an interaction between two
species occurring and is a means to identify links that are not
ecologically plausible, \emph{i.e.,} forbidden links (Jordano, 2016b),
or provide an idea of the \emph{complete} diet of a species (Strydom et
al., 2023). In contrast realised networks are highly localised and links
between species are contingent on both the co-occurrence of species as
well as the influence of the environment, and population and community
dynamics on predator choice. In the context of definitions and semantics
the links that are represented by a metaweb and a realised network are
different; links that are absent in a metaweb can be treated as being
truly absent, however links that are absent in a realised network cannot
be considered to be truly absent but are rather as absent due to the
broader environmental/community context. Importantly, a realised network
is \emph{not} simply the downscaling of a metaweb to a smaller scale
(\emph{e.g.,} moving from the country to the 1x1 km\textsuperscript{2}
scale based on fine-scale species co-occurrence) but represents a shift
towards capturing the higher level processes that determine the
\emph{realisation} of an interaction, \emph{i.e.,} the definition of an
edges shifts from being determined by interaction feasibility to that of
energetic choices/consequences. Thus, different network representations
are determined and constrained by different sets of assumptions as to
what the processes are that determine the presence/absence of an
interaction between two species as well as the resulting network
structure.

\section{From Nodes and Edges to Scale, Context, and
Process}\label{sec-process}

The interplay between network representation and network (node and edge)
definition is primarily governed by the process(es) that determine the
interaction between species, however these processes are also scale and
context dependent. Here we start by introducing the five core processes
that determine either the feasibility or the realisation of
interactions, namely: evolutionary compatibility, co-occurrence,
abundance, predator choice, and non-trophic interactions; while
simultaneously contextualising them within, and linking them to the
different network representations Figure~\ref{fig-process}. We can think
of the different network representations to be conceptually analogous to
the fundamental and realised niche, whereby the metaweb represents the
`fundamental diet niche' of a species and a realised network represents
the `realised diet' of a species. Of course these processes do not
function in a vacuum and do interact with/influence one another, but it
is still beneficial to present them in a categorical manner as these
different processes are often the underpinning logic in the development
of prediction/network models, the criteria for data collection in the
field, and the scale of organisation for which they are relevant
(species, population, or community).

\begin{figure}

\centering{

\includegraphics{images/representations.png}

}

\caption{\label{fig-process}Aligning the various processes that
determine interactions with the different network representations. First
we start with a `global metaweb' this network which captures all
possible interactions for an arbitrary collection of species, we can
further refine this network by taking in to consideration the
co-occurrence of these difference species - as shown here we have two
regions with some species (blue) that are found in both regions and
others endemic to either region one (pink) or region two (orange). These
regional metawebs to capture all possible interactions, however it only
considers species that co-occur. However even within a region we do not
expect all interactions to be realised but rather that there are
multiple configurations of the regional metaweb over both space and
time. The `state' of the different network realisations are ultimately
influenced not just by the co-occurrence of a species pair but rather
the larger community context such as the abundance of different species,
maximising energy gain, or indirect/higher order interactions.}

\end{figure}%

\subsection{The processes that determine species
interactions}\label{the-processes-that-determine-species-interactions}

\textbf{Evolutionary compatibility}

There is compelling evidence that an interaction occurring between two
species is the result of their shared (co)evolutionary history (Dalla
Riva \& Stouffer, 2016; Gómez et al., 2010; Segar et al., 2020) which,
in the more proximal sense, is manifested as the `trait complementarity'
between two species, whereby one species (the predator) has the
`correct' set of traits that allow it to chase, capture, kill, and
consume the other species (the prey). For species pairs where this
condition is not met the link is deemed to be forbidden (Jordano,
2016b); \emph{i.e.,} not physically possible and will always be absent
within a network. A network constructed on the basis of evolutionary
compatible links is most closely aligned with a metaweb, although it
would not be required that the species co-occur (as shown in
Figure~\ref{fig-process}), and arguably makes for a good approximation
of the `Eltonian niche' of species (Soberón, 2007). Finally, one should
be aware that it is possible to represent evolutionary compatible
interactions as either binary (possible vs forbidden) or as a
probability (Banville et al., 2024), where the probability represents
how likely the interaction between two species is to be possible.

\textbf{(Co)occurrence}

Although the outright assumption that because two species are
co-occurring it must mean that they are interacting is flawed (Blanchet
et al., 2020), it is of course impossible for two species to interact
(at least in terms of feeding links) if they are not co-occurring in
time and space. Thus, although co-occurrence data alone is insufficient
to build an accurate and ecologically meaningful representation of
\emph{feeding links} it is still a critical process that determines the
realisation of feeding links and allows us to constrain a global metaweb
to only consider `realised' communities (Dansereau et al., 2024) and an
understanding of the intersection of species interactions and their
co-occurrence is meaningful when one is operating in the space of trying
to determine the distribution of a species (Higino et al., 2023; Pollock
et al., 2014).

\textbf{Abundance}

The abundance of different the species within the community is thought
to influence the realisation of feeding links primarily in two ways.
Firstly there is the argument that that structure of networks (and the
interactions that they are composed of) are driven \emph{only} by the
abundance of the different species and that interactions are not
contingent on there being any compatibility (trait matching) between
them, \emph{sensu} neutral processes (Canard et al., 2012; Momal et al.,
2020). However, a more ecologically sound assumption would be that the
abundance of different prey species will influence which prey are
targeted or preferred by the predator as abundance influences factors
such as the likelihood of species meeting (Banville et al., 2024; Poisot
et al., 2015), or in the dynamic sense will influence the persistence of
viable populations.

\textbf{Predator choice (energetic cost)}

Ultimately, predator choice is underpinned by the energetic cost-benefit
of trying to catch, kill, and consume prey (where a predator will
optimise energy while minimising handling and search time), and is well
described within both optimal foraging (Pyke, 1984) and metabolic theory
(Brown et al., 2004). The energetic cost of feeding is itself influenced
by the interplay of both the energy content {[}\emph{i.e.,} body size;
Yodzis \& Innes (1992){]} as well as the density (abundance) of prey (as
this influences search time) and as a process will influence which links
are realised Figure~\ref{fig-process}. Additional work on on
understanding the energetic cost that the environment imposes on an
individual (Cherif et al., 2024) as well as the way a predator uses the
landscape to search for prey (Pawar et al., 2012) is bringing us closer
to accounting for the energetic cost of realising feeding links.

\textbf{Non-trophic interactions}

Perhaps not as intuitive when thinking about the processes that
determine feeding links (trophic interactions) is thinking about the
role of the ability of non-trophic interactions to modify either the
realisation or strength of trophic interactions (Golubski \& Abrams,
2011; Pilosof et al., 2017). Non-trophic interactions can modify
interactions either `directly' \emph{e.g.,} predator \emph{a}
outcompetes predator \emph{b} or `indirectly' \emph{e.g.,}
mutualistic/facilitative interactions will alter the fine-scale
distribution and abundance of species as well as their persistence
(Buche et al., 2024; Kéfi et al., 2012, 2015). The `unobservable' nature
of non-trophic interactions makes them a challenge to quantify, however
their importance in network dynamics should not be overlooked
(Staniczenko et al., 2010)

\subsection{Contextualising the processes that determine species
interactions}\label{contextualising-the-processes-that-determine-species-interactions}

It should be self evident that the different processes discussed above
are all ultimately going to influence the realisation of interactions as
well as the structure of a network, however they are acting at different
scales of organisation. Both the \textbf{co-occurrence} and the
\textbf{evolutionary compatibility} are valid at the scale of the
species pair of interest, that is the \emph{possibility} of an
interaction being present/absent is assessed at the pairwise level and
one is left with a `list' of interactions that are present/absent.
Although it is possible to build a network (\emph{i.e.,} metaweb) from
this information it is important to be aware that the structure of this
network is not constrained by real-world dynamics or conditions, and so
just because species are able to interact does not mean that they will
(Poisot et al., 2015). In order to construct a network who's structure
is a closer approximation of reality (localised interactions) one needs
to take into consideration the properties of the community as a whole
and not just the two species of interest, which requires more data at
the community scale, such as the abundance of species.

\section{Network construction is
nuanced}\label{network-construction-is-nuanced}

The act of constructing a `real world' network will ultimately be
delimited by its intended use, however the reality is that the empirical
collection of interaction data is both costly and challenging to execute
(Jordano, 2016a, 2016b), especially if one wants to capture \emph{all}
aspects of the processes discussed in Section~\ref{sec-process} (owing
to the different time and spatial scales they may be operating at). Thus
we often turn to models to either predict networks (be that the
interaction between two species, or network structure (Strydom et al.,
2021)), or as a means to identify missing interactions (gap fill) within
an existing empirical dataset (Biton et al., 2024; Dallas et al., 2017;
Stock, 2021), and so for the purpose of this discussion network
construction will be synonymous with using a model as a means to
represent or predict a network. That is not to say that there is no need
for empirical data collection but rather that using a model for food web
prediction (or reconstruction) is a more feasible approach as it allows
us to make inferences about interactions that are not happening in the
`observable now' (Strydom et al., 2021), with the added benefit that one
is able to build some uncertainty into the resulting network (Banville
et al., 2024). Additionally different models have different underlying
philosophies that allow us to capture one or a few of the processes
discussed in Section~\ref{sec-process}, and although the delimits and
defines what inferences can be made from the resulting network it also
allows us to isolate and understand how different processes determine
interactions (Song \& Levine, 2024; Stouffer, 2019). Here we will
introduce the three different types of network representations
(metawebs, realised networks, and structural networks), how they link
back to (and encode) the different processes determining interactions
Figure~\ref{fig-process}, and broadly discuss some of the modelling
approaches that are used to construct these different network types.
This is paralleled by a hypothetical case study (Box 1) where we
showcase the utility/applicability of the different network
representations in the context of trying to understand the feeding
dynamics of a seasonal community.

\begin{tcolorbox}[enhanced jigsaw, opacitybacktitle=0.6, left=2mm, title=\textcolor{quarto-callout-note-color}{\faInfo}\hspace{0.5em}{Box 1 - Why we need to aggregate networks at different scales: A
hypothetical case study}, arc=.35mm, colbacktitle=quarto-callout-note-color!10!white, leftrule=.75mm, toprule=.15mm, breakable, toptitle=1mm, opacityback=0, coltitle=black, titlerule=0mm, colback=white, rightrule=.15mm, colframe=quarto-callout-note-color-frame, bottomtitle=1mm, bottomrule=.15mm]

Although it might seem most prudent to be predicting, constructing, and
defining networks that are the closest representation of reality there
are pros and cons of constructing both realised networks as well as
metawebs. Let us take for example a community across time/through
seasons. In this community we expect species to be either present or
absent depending on the season (\emph{i.e.,} changes in co-occurrence)
as well as some species exhibiting seasonal diet shifts, these details
would be lost at the scale of the metaweb an it would be valuable to
construct either smaller metawebs for the different seasonal communities
(thereby capturing the changes in community diversity), or realised
networks for each season (to capture diet or ecosystem process shifts).
However, these small-scale networks lack the context of the bigger
picture that is available at the metaweb - that is it gives us a more
holistic idea of the entire diet range of a specific species, which is
important when one needs to make conservation-based/applied decisions
(\emph{e.g.,} conserving the entire diet of a species and not just
seasonal prey items) as well as providing information on interactions
that may be possible regardless of the environmental/community context
(species may have the capacity to consume certain prey items but do not
do so due to local conditions). With this is mind let us see how the
different network aggregations can be used

\textbf{1: A global metaweb}

Knowledge of the entire diet breadth of a species is valuable especially
in terms of understanding how a species will respond to changes in the
community - \emph{e.g.,} invasions/rewilding exercises (where does the
new species `fit' within the network?) as well as potential capacity to
shift its diet. ALthough this might make sense across space and not time
but certain species act as links across the landscape.

\textbf{2: A seasonal metaweb}

Knowledge at the finer scale is also valuable to understand/identify
that there are in fact differences between the seasons

\textbf{3: A seasonal realised network}

Dynamics are useful because they are a representation of the different
configurations/energy flows/ecosystem processes. Also to detect more
nuanced shifts in diet - \emph{e.g.,} seasonal diet shifts.

\textbf{4: A structural network}

\textbf{Data trade off}

Above we highlight the practical uses of the different network
configurations but we also need to take into consideration the barriers
to construction/associated data needs/cost and acknowledge them.
Basically in the ideal world we would have all this information at hand
but in reality we might be sitting with seasonal metawebs\ldots{}

\end{tcolorbox}

\subsection{Models that predict metawebs (feasible
interactions)}\label{models-that-predict-metawebs-feasible-interactions}

This is perhaps the most developed group of models; with a variety of
approaches having been developed that typically determine the
feasibility of an interaction using the trait compatibility between
predator and prey (\emph{i.e.} their evolutionary compatibility) to
determine `feeding rules' (Morales-Castilla et al., 2015). These feeding
rules are broadly elucidated in two different ways; mechanistic feeding
rules can be explicitly defined and applied to a community (Dunne et
al., 2008; \emph{e.g.,} Shaw et al., 2024) or they are inferred from a
community for which there are interaction data and the `rules' are then
applied to a different community (Caron et al., 2022; Cirtwill et al.,
2019; Desjardins-Proulx et al., 2017; Eklöf et al., 2013; Llewelyn et
al., 2023; Pichler et al., 2020; Strydom et al., 2022; \emph{e.g.,}
Strydom et al., 2023). The fundamental difference between these two
model groups is that `mechanistic models' rely on expert knowledge and
make explicit assumptions on trait-feeding relationships, whereas the
`pattern finding' models are dependent on existing datasets from which
to elucidate feeding rules. These models are useful for determining all
feasible interactions for a specific community, and owing to the
availability of empirical interaction datasets (Gray et al., 2015;
\emph{e.g.,} Poelen et al., 2014; Poisot, Baiser, et al., 2016), as well
as the development of model testing/benchmarking tools (Poisot, 2023),
means that these models can be validated and (with relative confidence)
be used to construct first draft networks for communities for which we
have no interaction data (Strydom et al., 2022), and are valuable not
only in data poor regions but also for predicting interactions for
`unobservable' communities \emph{e.g.,} prehistoric networks (Fricke et
al., 2022; Yeakel et al., 2014) or future, novel community assemblages.
Importantly metawebs are inherently `static' in the sense that they are
\emph{not} able to capture dynamic processes (since the notion of
feasibility is all or nothing), however they provide a bigger picture
context (\emph{e.g.,} understanding the \emph{entire} diet breadth of a
species) and often require little data to construct.

\subsection{Models that predict realised networks (realised
interactions)}\label{models-that-predict-realised-networks-realised-interactions}

In order to construct realised networks models need to incorporate
\emph{both} the feasibility of interactions (\emph{i.e.,} determine the
entire diet breadth of a species) as well as then determine which
interactions are realised (\emph{i.e.,} incorporate the `cost' of
interactions). As far as we are aware there is no model that explicitly
accounts for both of these `rules' and rather \emph{only} account for
processes that determine the realisation of an interaction (\emph{i.e.,}
abundance, predator choice, or non-trophic interactions). Although the
use of allometry \emph{i.e.,} body size (Beckerman et al., 2006;
\emph{e.g.,} Valdovinos et al., 2023) may represent a first step in
capturing `evolutionary compatibility' alongside more energy (predator
choice) driven processes we still need to account for other traits that
determine feeding compatibility (\emph{e.g.,} Van De Walle et al., 2023
show how incorporating prey defensive properties alongside body size
improves predictions). In terms of constructing realised networks, diet
models (Beckerman et al., 2006; Petchey et al., 2008) have been used
construct networks based on both predator choice (as determined by the
handling time, energy content, and predator attack rate) as well as
abundance (prey density) and progress has also been made in
understanding the compartmentation of energy in networks and how this
influences energy acquisition (Krause et al., 2003; Wootton et al.,
2023). As realised networks are are build on the concept of dynamic
processes (the abundance of species will always be in flux) these
networks are valuable for understanding the behaviour of networks over
time or their response to change (Curtsdotter et al., 2019; Delmas et
al., 2017; Lajaaiti et al., 2024). However, the are `costly' to
construct (requiring data about the entire community as it is the
behaviour of the system that determines the behaviour of the part) and
also lack the larger context afforded by metawebs.

\subsection{Models that predict structure (interaction
agnostic)}\label{models-that-predict-structure-interaction-agnostic}

Although we identify mechanisms that determine species interactions in
Section~\ref{sec-process} not all models that are used to predict
networks explicitly operate at the `process' level, but rather represent
the \emph{structure} of a network based on a series of \emph{a priori}
assumptions as to the distribution of links between species (typically
trophic not taxonomic species). These models operate by parametrising an
aspect of the network structure, (\emph{e.g.,} the niche model (Williams
\& Martinez, 2000) makes an assumption as to the expected connectance of
the network,although see Allesina \& Pascual (2009) for a parameter-free
model) or alternatively uses structural features of an exiting
\emph{realised} network (\emph{e.g.,} stochastic block model, Xie et al.
(2017)). Importantly these structural models do not make species
specific predictions (they are usually species agnostic and treat nodes
as trophic species) and so cannot be used to determine if an interaction
is either possible \emph{or} realised between two species (\emph{i.e.,}
one cannot use these models to determine if species \(a\) eats species
\(b\)). Although this means this suite of models are unsuitable as tools
for predicting species-specific interactions, they have been shown to be
sufficient tools to predict the structure of networks (Williams \&
Martinez, 2008), and provide a data-light (the models often only require
species richness) but assumption heavy (the resulting network structure
is determined by an assumption of network structure) way to construct a
network.

\section{Making Progress with
Networks}\label{making-progress-with-networks}

\subsection{Further development of models and
tools}\label{further-development-of-models-and-tools}

There has been a suite of models that have been developed to predict
feeding links, however we are lacking in tools that are explicitly
taking into consideration estimating both the feasibility as well as
realisation of links, \emph{i.e.,} both interactions and structure
simultaneously (Strydom et al., 2021). This could be addressed either
through the development of tools that do both (predict both interactions
and structure), or to develop an ensemble modelling approach (Becker et
al., 2022; Terry \& Lewis, 2020) or tools that will allow for the
downsampling of metawebs into realised networks (\emph{e.g.,}
Roopnarine, 2006). Additionally although realised networks are more
closely aligned with capturing interaction strength we lack models that
allow us to quantify this (Strydom et al., 2021; Wells \& O'Hara, 2013).
In addition to the more intentional development of models we also need
to consider the validation of these models, there have been developments
and discussions for assessing how well a model recovers pairwise
interactions (Poisot, 2023; Strydom et al., 2021), although the rate of
false-negatives that may be present in the testing data still present a
challenge (Catchen et al., 2023), we still lack clear set of guidelines
for benchmarking the ability of models to recover structure (Allesina et
al., 2008).

\subsection{At what scale should we be predicting and using
networks?}\label{at-what-scale-should-we-be-predicting-and-using-networks}

We lack an understanding of which processes drive the differences
between different scales (Saravia et al., 2022), as well as to what the
appropriate level of aggregation is for a `network' (Estay et al.,
2023). Which presents a challenge both in deciding what the appropriate
spatial (which influences both network properties (Galiana et al.,
2018), as well as dynamics (Fortin et al., 2021; Rooney et al., 2008)),
and time scales (\emph{e.g.,} accounting for seasonal turnover in
communities (Brimacombe et al., 2021; Laender et al., 2010) and
different timescales of co-occurrence records (Brimacombe et al., 2024))
are for constructing not only a network but also which type of network
representation. Although multilayer networks may allow us to encode the
nuances of space and time (Hutchinson et al., 2019) we still need to
understand the implications of \emph{e.g.,} constructing networks that
are not at ecologically but rather politically relevant scales (Strydom
et al., 2022) and what we can learn or infer from networks a these
scales.

\section{The future value of
networks}\label{the-future-value-of-networks}

\begin{quote}
developing a dictionary of use\ldots{} that helps navigate between the
levels and assumptions
\end{quote}

It should be clear that there is a high degree of interrelatedness and
overlap between the way a network is constructed (modelled or predicted)
and the process(es) it captures, these are encoded (embedded) within the
network representation and ultimately influences how the network can and
should be used (Berlow et al., 2008; Petchey et al., 2011). It is
probably both this nuance as well as a lack of clear boundaries and
guidelines as to the links between network form and function (although
see Delmas et al., 2019) that has stifled the `productive use' of
networks beyond inventorying the interactions between species. Although,
progress with using networks as a means to address questions within
larger bodies of ecological theory \emph{e.g.,} invasion biology (Hui \&
Richardson, 2019) and co-existence theory (García-Callejas et al.,
2023), has been made we still need to have a discussion on what the
appropriate network representation for the task at hand would be. This
is highlighted in Box 1, and underscores that we need to evaluate
exactly what process a specific network representation captures as well
as its suitability for the question of interest.

\begin{longtable}[]{@{}
  >{\raggedright\arraybackslash}p{(\columnwidth - 4\tabcolsep) * \real{0.2192}}
  >{\raggedright\arraybackslash}p{(\columnwidth - 4\tabcolsep) * \real{0.3699}}
  >{\raggedright\arraybackslash}p{(\columnwidth - 4\tabcolsep) * \real{0.4110}}@{}}
\caption{An informative table}\tabularnewline
\toprule\noalign{}
\begin{minipage}[b]{\linewidth}\raggedright
Question (broad)
\end{minipage} & \begin{minipage}[b]{\linewidth}\raggedright
Question (specific)
\end{minipage} & \begin{minipage}[b]{\linewidth}\raggedright
Network representation
\end{minipage} \\
\midrule\noalign{}
\endfirsthead
\toprule\noalign{}
\begin{minipage}[b]{\linewidth}\raggedright
Question (broad)
\end{minipage} & \begin{minipage}[b]{\linewidth}\raggedright
Question (specific)
\end{minipage} & \begin{minipage}[b]{\linewidth}\raggedright
Network representation
\end{minipage} \\
\midrule\noalign{}
\endhead
\bottomrule\noalign{}
\endlastfoot
Species invasions & What species will the invading species interact
with? & Regional metaweb but need to derive information from a global
metaweb since these are interactions that are `novel' \\
Species invasions & How does the invading species alter network dynamics
and function? & Realised network (after having moved through the global
metaweb to understand which interactions are feasible) \\
Range shifts and novel communities & Under global change how will novel
community assemblages interact? & Global metaweb, need context of
broader community \\
Extinctions & Cascading effect of the loss of a species from the network
& Regional metaweb - need to account for entire diet, a realised network
will exclude the entire diet but will allow to elucidate the final
structure \\
Species/community persistence & Dynamics over time.
Stability/resilience. How does a change in pop \emph{A} affect pop
\emph{B}? & Realised networks - but dynamic! \\
Synthetic networks & Creating ecologically plausible communities for
synthetic analyses & Structural networks - data light! \\
Practical use & What is both attainable (data constraints) but also of
practical use to `real world' decision making. So moving from theory to
applied & ??Regional metawebs?? \\
\end{longtable}

\section*{References}\label{references}
\addcontentsline{toc}{section}{References}

\phantomsection\label{refs}
\begin{CSLReferences}{1}{0}
\bibitem[\citeproctext]{ref-allesinaGeneralModelFood2008}
Allesina, S., Alonso, D., \& Pascual, M. (2008). A {General Model} for
{Food Web Structure}. \emph{Science}, \emph{320}(5876), 658--661.
\url{https://doi.org/10.1126/science.1156269}

\bibitem[\citeproctext]{ref-allesinaFoodWebModels2009}
Allesina, S., \& Pascual, M. (2009). Food web models: A plea for groups.
\emph{Ecology Letters}, \emph{12}(7), 652--662.
\url{https://doi.org/10.1111/j.1461-0248.2009.01321.x}

\bibitem[\citeproctext]{ref-banvilleDecipheringProbabilisticSpecies2024}
Banville, F., Strydom, T., Blyth, P., Brimacombe, C., Catchen, M. D.,
Dansereau, G., Higino, G., Malpas, T., Mayall, H., Norman, K., Gravel,
D., \& Poisot, T. (2024). \emph{Deciphering probabilistic species
interaction networks}. EcoEvoRxiv. \url{https://doi.org/10.32942/X28G8Z}

\bibitem[\citeproctext]{ref-beckerOptimisingPredictiveModels2022}
Becker, D. J., Albery, G. F., Sjodin, A. R., Poisot, T., Bergner, L. M.,
Chen, B., Cohen, L. E., Dallas, T. A., Eskew, E. A., Fagre, A. C.,
Farrell, M. J., Guth, S., Han, B. A., Simmons, N. B., Stock, M.,
Teeling, E. C., \& Carlson, C. J. (2022). Optimising predictive models
to prioritise viral discovery in zoonotic reservoirs. \emph{The Lancet
Microbe}, \emph{3}(8), e625--e637.
\url{https://doi.org/10.1016/S2666-5247(21)00245-7}

\bibitem[\citeproctext]{ref-beckermanForagingBiologyPredicts2006}
Beckerman, A. P., Petchey, O. L., \& Warren, P. H. (2006). Foraging
biology predicts food web complexity. \emph{Proceedings of the National
Academy of Sciences}, \emph{103}(37), 13745--13749.
\url{https://doi.org/10.1073/pnas.0603039103}

\bibitem[\citeproctext]{ref-berlowGoldilocksFactorFood2008}
Berlow, E. L., Brose, U., \& Martinez, N. D. (2008). The {``{Goldilocks}
factor''} in food webs. \emph{Proceedings of the National Academy of
Sciences}, \emph{105}(11), 4079--4080.
\url{https://doi.org/10.1073/pnas.0800967105}

\bibitem[\citeproctext]{ref-berlowInteractionStrengthsFood2004}
Berlow, E. L., Neutel, A.-M., Cohen, J. E., de Ruiter, P. C., Ebenman,
B., Emmerson, M., Fox, J. W., Jansen, V. A. A., Iwan Jones, J.,
Kokkoris, G. D., Logofet, D. O., McKane, A. J., Montoya, J. M., \&
Petchey, O. (2004). Interaction strengths in food webs: Issues and
opportunities. \emph{Journal of Animal Ecology}, \emph{73}(3), 585--598.
\url{https://doi.org/10.1111/j.0021-8790.2004.00833.x}

\bibitem[\citeproctext]{ref-bitonInductiveLinkPrediction2024}
Biton, B., Puzis, R., \& Pilosof, S. (2024). \emph{Inductive link
prediction boosts data availability and enables cross-community link
prediction in ecological networks}.

\bibitem[\citeproctext]{ref-blanchetCooccurrenceNotEvidence2020}
Blanchet, F. G., Cazelles, K., \& Gravel, D. (2020). Co-occurrence is
not evidence of ecological interactions. \emph{Ecology Letters},
\emph{23}(7), 1050--1063. \url{https://doi.org/10.1111/ele.13525}

\bibitem[\citeproctext]{ref-brimacombeInferredSeasonalInteraction2021}
Brimacombe, C., Bodner, K., \& Fortin, M.-J. (2021). Inferred seasonal
interaction rewiring of a freshwater stream fish network.
\emph{Ecography}, \emph{44}(2), 219--230.
\url{https://doi.org/10.1111/ecog.05452}

\bibitem[\citeproctext]{ref-brimacombeApplyingMethodIts2024}
Brimacombe, C., Bodner, K., \& Fortin, M.-J. (2024). \emph{Applying a
method before its proof-of-concept: {A} cautionary tale using inferred
food webs}. \url{https://doi.org/10.13140/RG.2.2.22076.65927}

\bibitem[\citeproctext]{ref-brimacombeShortcomingsReusingSpecies2023}
Brimacombe, C., Bodner, K., Michalska-Smith, M., Poisot, T., \& Fortin,
M.-J. (2023). Shortcomings of reusing species interaction networks
created by different sets of researchers. \emph{PLOS Biology},
\emph{21}(4), e3002068.
\url{https://doi.org/10.1371/journal.pbio.3002068}

\bibitem[\citeproctext]{ref-brownMetabolicTheoryEcology2004}
Brown, J. H., Gillooly, J. F., Allen, A. P., Savage, V. M., \& West, G.
B. (2004). Toward a {Metabolic Theory} of {Ecology}. \emph{Ecology},
\emph{85}(7), 1771--1789. \url{https://doi.org/10.1890/03-9000}

\bibitem[\citeproctext]{ref-bucheMultitrophicHigherOrderInteractions2024}
Buche, L., Bartomeus, I., \& Godoy, O. (2024). Multitrophic
{Higher-Order Interactions Modulate Species Persistence}. \emph{The
American Naturalist}, \emph{203}(4), 458--472.
\url{https://doi.org/10.1086/729222}

\bibitem[\citeproctext]{ref-canardEmergenceStructuralPatterns2012}
Canard, E., Mouquet, N., Marescot, L., Gaston, K. J., Gravel, D., \&
Mouillot, D. (2012). Emergence of {Structural Patterns} in {Neutral
Trophic Networks}. \emph{PLOS ONE}, \emph{7}(8), e38295.
\url{https://doi.org/10.1371/journal.pone.0038295}

\bibitem[\citeproctext]{ref-caronTraitmatchingModelsPredict2024}
Caron, D., Brose, U., Lurgi, M., Blanchet, F. G., Gravel, D., \&
Pollock, L. J. (2024). Trait-matching models predict pairwise
interactions across regions, not food web properties. \emph{Global
Ecology and Biogeography}, \emph{33}(4), e13807.
\url{https://doi.org/10.1111/geb.13807}

\bibitem[\citeproctext]{ref-caronAddressingEltonianShortfall2022}
Caron, D., Maiorano, L., Thuiller, W., \& Pollock, L. J. (2022).
Addressing the {Eltonian} shortfall with trait-based interaction models.
\emph{Ecology Letters}, \emph{25}(4), 889--899.
\url{https://doi.org/10.1111/ele.13966}

\bibitem[\citeproctext]{ref-catchenMissingLinkDiscerning2023}
Catchen, M. D., Poisot, T., Pollock, L. J., \& Gonzalez, A. (2023).
\emph{The missing link: Discerning true from false negatives when
sampling species interaction networks}.

\bibitem[\citeproctext]{ref-cherifEnvironmentRescueCan2024}
Cherif, M., Brose, U., Hirt, M. R., Ryser, R., Silve, V., Albert, G.,
Arnott, R., Berti, E., Cirtwill, A., Dyer, A., Gauzens, B., Gupta, A.,
Ho, H.-C., Portalier, S. M. J., Wain, D., \& Wootton, K. (2024). The
environment to the rescue: Can physics help predict predator--prey
interactions? \emph{Biological Reviews}, \emph{n/a}(n/a).
\url{https://doi.org/10.1111/brv.13105}

\bibitem[\citeproctext]{ref-cirtwillQuantitativeFrameworkInvestigating2019}
Cirtwill, A. R., Eklf, A., Roslin, T., Wootton, K., \& Gravel, D.
(2019). A quantitative framework for investigating the reliability of
empirical network construction. \emph{Methods in Ecology and Evolution},
\emph{10}(6), 902--911. \url{https://doi.org/10.1111/2041-210X.13180}

\bibitem[\citeproctext]{ref-cleggImpactIntraspecificVariation2018}
Clegg, T., Ali, M., \& Beckerman, A. P. (2018). The impact of
intraspecific variation on food web structure. \emph{Ecology},
\emph{99}(12), 2712--2720. \url{https://doi.org/10.1002/ecy.2523}

\bibitem[\citeproctext]{ref-curtsdotterEcosystemFunctionPredator2019}
Curtsdotter, A., Banks, H. T., Banks, J. E., Jonsson, M., Jonsson, T.,
Laubmeier, A. N., Traugott, M., \& Bommarco, R. (2019). Ecosystem
function in predator--prey food webs---confronting dynamic models with
empirical data. \emph{Journal of Animal Ecology}, \emph{88}(2),
196--210. \url{https://doi.org/10.1111/1365-2656.12892}

\bibitem[\citeproctext]{ref-dallarivaExploringEvolutionarySignature2016}
Dalla Riva, G. V., \& Stouffer, D. B. (2016). Exploring the evolutionary
signature of food webs' backbones using functional traits. \emph{Oikos},
\emph{125}(4), 446--456. \url{https://doi.org/10.1111/oik.02305}

\bibitem[\citeproctext]{ref-dallasPredictingCrypticLinks2017}
Dallas, T., Park, A. W., \& Drake, J. M. (2017). Predicting cryptic
links in host-parasite networks. \emph{PLOS Computational Biology},
\emph{13}(5), e1005557.
\url{https://doi.org/10.1371/journal.pcbi.1005557}

\bibitem[\citeproctext]{ref-dansereauSpatiallyExplicitPredictions2024}
Dansereau, G., Barros, C., \& Poisot, T. (2024). Spatially explicit
predictions of food web structure from regional-level data.
\emph{Philosophical Transactions of the Royal Society B: Biological
Sciences}, \emph{379}(1909).
\url{https://doi.org/10.1098/rstb.2023.0166}

\bibitem[\citeproctext]{ref-delmasAnalysingEcologicalNetworks2019}
Delmas, E., Besson, M., Brice, M.-H., Burkle, L. A., Riva, G. V. D.,
Fortin, M.-J., Gravel, D., Guimarães, P. R., Hembry, D. H., Newman, E.
A., Olesen, J. M., Pires, M. M., Yeakel, J. D., \& Poisot, T. (2019).
Analysing ecological networks of species interactions. \emph{Biological
Reviews}, \emph{94}(1), 16--36. \url{https://doi.org/10.1111/brv.12433}

\bibitem[\citeproctext]{ref-delmasSimulationsBiomassDynamics2017}
Delmas, E., Brose, U., Gravel, D., Stouffer, D. B., \& Poisot, T.
(2017). Simulations of biomass dynamics in community food webs.
\emph{Methods in Ecology and Evolution}, \emph{8}(7), 881--886.
\url{https://doi.org/10.1111/2041-210X.12713}

\bibitem[\citeproctext]{ref-desjardins-proulxEcologicalInteractionsNetflix2017}
Desjardins-Proulx, P., Laigle, I., Poisot, T., \& Gravel, D. (2017).
Ecological interactions and the {Netflix} problem. \emph{PeerJ},
\emph{5}, e3644. \url{https://doi.org/10.7717/peerj.3644}

\bibitem[\citeproctext]{ref-dunneNetworkStructureFood2006}
Dunne, J. A. (2006). The {Network Structure} of {Food Webs}. In J. A.
Dunne \& M. Pascual (Eds.), \emph{Ecological networks: {Linking}
structure and dynamics} (pp. 27--86). Oxford University Press.

\bibitem[\citeproctext]{ref-dunneCompilationNetworkAnalyses2008}
Dunne, J. A., Williams, R. J., Martinez, N. D., Wood, R. A., \& Erwin,
D. H. (2008). Compilation and {Network Analyses} of {Cambrian Food
Webs}. \emph{PLOS Biology}, \emph{6}(4), e102.
\url{https://doi.org/10.1371/journal.pbio.0060102}

\bibitem[\citeproctext]{ref-eklofSecondaryExtinctionsFood2013}
Eklöf, A., Tang, S., \& Allesina, S. (2013). Secondary extinctions in
food webs: A {Bayesian} network approach. \emph{Methods in Ecology and
Evolution}, \emph{4}(8), 760--770.
\url{https://doi.org/10.1111/2041-210X.12062}

\bibitem[\citeproctext]{ref-estayEditorialPatternsProcesses2023}
Estay, S. A., Fortin, M.-J., \& López, D. N. (2023). Editorial:
{Patterns} and processes in ecological networks over space.
\emph{Frontiers in Ecology and Evolution}, \emph{11}.

\bibitem[\citeproctext]{ref-fortinNetworkEcologyDynamic2021}
Fortin, M.-J., Dale, M. R. T., \& Brimacombe, C. (2021). Network ecology
in dynamic landscapes. \emph{Proceedings of the Royal Society B:
Biological Sciences}, \emph{288}(1949), rspb.2020.1889, 20201889.
\url{https://doi.org/10.1098/rspb.2020.1889}

\bibitem[\citeproctext]{ref-frickeCollapseTerrestrialMammal2022}
Fricke, E. C., Hsieh, C., Middleton, O., Gorczynski, D., Cappello, C.
D., Sanisidro, O., Rowan, J., Svenning, J.-C., \& Beaudrot, L. (2022).
Collapse of terrestrial mammal food webs since the {Late Pleistocene}.
\emph{Science}, \emph{377}(6609), 1008--1011.
\url{https://doi.org/10.1126/science.abn4012}

\bibitem[\citeproctext]{ref-galianaSpatialScalingSpecies2018}
Galiana, N., Lurgi, M., Claramunt-López, B., Fortin, M.-J., Leroux, S.,
Cazelles, K., Gravel, D., \& Montoya, J. M. (2018). The spatial scaling
of species interaction networks. \emph{Nature Ecology \& Evolution},
\emph{2}(5), 782--790. \url{https://doi.org/10.1038/s41559-018-0517-3}

\bibitem[\citeproctext]{ref-garcia-callejasNonrandomInteractionsGuilds2023}
García-Callejas, D., Godoy, O., Buche, L., Hurtado, M., Lanuza, J. B.,
Allen-Perkins, A., \& Bartomeus, I. (2023). Non-random interactions
within and across guilds shape the potential to coexist in multi-trophic
ecological communities. \emph{Ecology Letters}, \emph{26}(6), 831--842.
\url{https://doi.org/10.1111/ele.14206}

\bibitem[\citeproctext]{ref-golubskiModifyingModifiersWhat2011}
Golubski, A. J., \& Abrams, P. A. (2011). Modifying modifiers: What
happens when interspecific interactions interact? \emph{Journal of
Animal Ecology}, \emph{80}(5), 1097--1108.
\url{https://doi.org/10.1111/j.1365-2656.2011.01852.x}

\bibitem[\citeproctext]{ref-gomezEcologicalInteractionsAre2010}
Gómez, J. M., Verdú, M., \& Perfectti, F. (2010). Ecological
interactions are evolutionarily conserved across the entire tree of
life. \emph{Nature}, \emph{465}(7300), 918--921.
\url{https://doi.org/10.1038/nature09113}

\bibitem[\citeproctext]{ref-grayJoiningDotsAutomated2015}
Gray, C., Figueroa, D. H., Hudson, L. N., Ma, A., Perkins, D., \&
Woodward, G. (2015). Joining the dots: {An} automated method for
constructing food webs from compendia of published interactions.
\emph{Food Webs}, \emph{5}, 11--20.
\url{https://doi.org/10.1016/j.fooweb.2015.09.001}

\bibitem[\citeproctext]{ref-higinoMismatchIUCNRange2023}
Higino, G. T., Banville, F., Dansereau, G., Muñoz, N. R. F., Windsor,
F., \& Poisot, T. (2023). Mismatch between {IUCN} range maps and species
interactions data illustrated using the {Serengeti} food web.
\emph{PeerJ}, \emph{11}, e14620.
\url{https://doi.org/10.7717/peerj.14620}

\bibitem[\citeproctext]{ref-huiHowInvadeEcological2019}
Hui, C., \& Richardson, D. M. (2019). How to {Invade} an {Ecological
Network}. \emph{Trends in Ecology \& Evolution}, \emph{34}(2), 121--131.
\url{https://doi.org/10.1016/j.tree.2018.11.003}

\bibitem[\citeproctext]{ref-hutchinsonSeeingForestTrees2019}
Hutchinson, M. C., Bramon Mora, B., Pilosof, S., Barner, A. K., Kéfi,
S., Thébault, E., Jordano, P., \& Stouffer, D. B. (2019). Seeing the
forest for the trees: {Putting} multilayer networks to work for
community ecology. \emph{Functional Ecology}, \emph{33}(2), 206--217.
\url{https://doi.org/10.1111/1365-2435.13237}

\bibitem[\citeproctext]{ref-jordanoChasingEcologicalInteractions2016}
Jordano, P. (2016a). Chasing {Ecological Interactions}. \emph{PLOS
Biology}, \emph{14}(9), e1002559.
\url{https://doi.org/10.1371/journal.pbio.1002559}

\bibitem[\citeproctext]{ref-jordanoSamplingNetworksEcological2016}
Jordano, P. (2016b). Sampling networks of ecological interactions.
\emph{Functional Ecology}. \url{https://doi.org/10.1111/1365-2435.12763}

\bibitem[\citeproctext]{ref-kefiNetworkStructureFood2015}
Kéfi, S., Berlow, E. L., Wieters, E. A., Joppa, L. N., Wood, S. A.,
Brose, U., \& Navarrete, S. A. (2015). Network structure beyond food
webs: Mapping non-trophic and trophic interactions on {Chilean} rocky
shores. \emph{Ecology}, \emph{96}(1), 291--303.
\url{https://doi.org/10.1890/13-1424.1}

\bibitem[\citeproctext]{ref-kefiMoreMealIntegrating2012}
Kéfi, S., Berlow, E. L., Wieters, E. A., Navarrete, S. A., Petchey, O.
L., Wood, S. A., Boit, A., Joppa, L. N., Lafferty, K. D., Williams, R.
J., Martinez, N. D., Menge, B. A., Blanchette, C. A., Iles, A. C., \&
Brose, U. (2012). More than a meal{\ldots{}} integrating non-feeding
interactions into food webs: {More} than a meal {\ldots{}}.
\emph{Ecology Letters}, \emph{15}(4), 291--300.
\url{https://doi.org/10.1111/j.1461-0248.2011.01732.x}

\bibitem[\citeproctext]{ref-krauseCompartmentsRevealedFoodweb2003}
Krause, A. E., Frank, K. A., Mason, D. M., Ulanowicz, R. E., \& Taylor,
W. W. (2003). Compartments revealed in food-web structure.
\emph{Nature}, \emph{426}(6964), 282--285.
\url{https://doi.org/10.1038/nature02115}

\bibitem[\citeproctext]{ref-laenderCarbonTransferHerbivore2010}
Laender, F. D., Oevelen, D. V., Soetaert, K., \& Middelburg, J. J.
(2010). Carbon transfer in a herbivore- and microbial loop-dominated
pelagic food webs in the southern {Barents Sea} during spring and
summer. \emph{Marine Ecology Progress Series}, \emph{398}, 93--107.
\url{https://doi.org/10.3354/meps08335}

\bibitem[\citeproctext]{ref-lajaaitiEcologicalNetworksDynamicsJlJulia2024}
Lajaaiti, I., Bonnici, I., Kéfi, S., Mayall, H., Danet, A., Beckerman,
A. P., Malpas, T., \& Delmas, E. (2024).
\emph{{EcologicalNetworksDynamics}.jl {A Julia} package to simulate the
temporal dynamics of complex ecological networks} (p.
2024.03.20.585899). bioRxiv.
\url{https://doi.org/10.1101/2024.03.20.585899}

\bibitem[\citeproctext]{ref-lindemanTrophicDynamicAspectEcology1942}
Lindeman, R. L. (1942). The {Trophic-Dynamic Aspect} of {Ecology}.
\emph{Ecology}, \emph{23}(4), 399--417.
\url{https://doi.org/10.2307/1930126}

\bibitem[\citeproctext]{ref-llewelynPredictingPredatorPrey2023}
Llewelyn, J., Strona, G., Dickman, C. R., Greenville, A. C., Wardle, G.
M., Lee, M. S. Y., Doherty, S., Shabani, F., Saltré, F., \& Bradshaw, C.
J. A. (2023). Predicting predator--prey interactions in terrestrial
endotherms using random forest. \emph{Ecography}, \emph{2023}(9),
e06619. \url{https://doi.org/10.1111/ecog.06619}

\bibitem[\citeproctext]{ref-momalTreebasedInferenceSpecies2020}
Momal, R., Robin, S., \& Ambroise, C. (2020). Tree-based inference of
species interaction networks from abundance data. \emph{Methods in
Ecology and Evolution}, \emph{11}(5), 621--632.
\url{https://doi.org/10.1111/2041-210X.13380}

\bibitem[\citeproctext]{ref-morales-castillaInferringBioticInteractions2015}
Morales-Castilla, I., Matias, M. G., Gravel, D., \& Araújo, M. B.
(2015). Inferring biotic interactions from proxies. \emph{Trends in
Ecology \& Evolution}, \emph{30}(6), 347--356.
\url{https://doi.org/10.1016/j.tree.2015.03.014}

\bibitem[\citeproctext]{ref-pawarDimensionalityConsumerSearch2012}
Pawar, S., Dell, A. I., \& Savage, V. M. (2012). Dimensionality of
consumer search space drives trophic interaction strengths.
\emph{Nature}, \emph{486}(7404), 485--489.
\url{https://doi.org/10.1038/nature11131}

\bibitem[\citeproctext]{ref-petcheySizeForagingFood2008}
Petchey, O. L., Beckerman, A. P., Riede, J. O., \& Warren, P. H. (2008).
Size, foraging, and food web structure. \emph{Proceedings of the
National Academy of Sciences}, \emph{105}(11), 4191--4196.
\url{https://doi.org/10.1073/pnas.0710672105}

\bibitem[\citeproctext]{ref-petcheyFitEfficiencyBiology2011}
Petchey, O. L., Beckerman, A. P., Riede, J. O., \& Warren, P. H. (2011).
Fit, efficiency, and biology: {Some} thoughts on judging food web
models. \emph{Journal of Theoretical Biology}, \emph{279}(1), 169--171.
\url{https://doi.org/10.1016/j.jtbi.2011.03.019}

\bibitem[\citeproctext]{ref-pichlerMachineLearningAlgorithms2020}
Pichler, M., Boreux, V., Klein, A.-M., Schleuning, M., \& Hartig, F.
(2020). Machine learning algorithms to infer trait-matching and predict
species interactions in ecological networks. \emph{Methods in Ecology
and Evolution}, \emph{11}(2), 281--293.
\url{https://doi.org/10.1111/2041-210X.13329}

\bibitem[\citeproctext]{ref-pilosofMultilayerNatureEcological2017}
Pilosof, S., Porter, M. A., Pascual, M., \& Kéfi, S. (2017). The
multilayer nature of ecological networks. \emph{Nature Ecology \&
Evolution}, \emph{1}(4), 101.
\url{https://doi.org/10.1038/s41559-017-0101}

\bibitem[\citeproctext]{ref-poelenGlobalBioticInteractions2014}
Poelen, J. H., Simons, J. D., \& Mungall, C. J. (2014). Global biotic
interactions: {An} open infrastructure to share and analyze
species-interaction datasets. \emph{Ecological Informatics}, \emph{24},
148--159. \url{https://doi.org/10.1016/j.ecoinf.2014.08.005}

\bibitem[\citeproctext]{ref-poisotGuidelinesPredictionSpecies2023}
Poisot, T. (2023). Guidelines for the prediction of species interactions
through binary classification. \emph{Methods in Ecology and Evolution},
\emph{14}(5), 1333--1345. \url{https://doi.org/10.1111/2041-210X.14071}

\bibitem[\citeproctext]{ref-poisotMangalMakingEcological2016}
Poisot, T., Baiser, B., Dunne, J., Kéfi, S., Massol, F., Mouquet, N.,
Romanuk, T. N., Stouffer, D. B., Wood, S. A., \& Gravel, D. (2016).
Mangal -- making ecological network analysis simple. \emph{Ecography},
\emph{39}(4), 384--390. \url{https://doi.org/10.1111/ecog.00976}

\bibitem[\citeproctext]{ref-poisotStructureProbabilisticNetworks2016}
Poisot, T., Cirtwill, A., Cazelles, K., Gravel, D., Fortin, M.-J., \&
Stouffer, D. (2016). The structure of probabilistic networks.
\emph{Methods in Ecology and Evolution}, \emph{7}(3), 303--312.
\url{https://doi.org/10}

\bibitem[\citeproctext]{ref-poisotSpeciesWhyEcological2015}
Poisot, T., Stouffer, D. B., \& Gravel, D. (2015). Beyond species: Why
ecological interaction networks vary through space and time.
\emph{Oikos}, \emph{124}(3), 243--251.
\url{https://doi.org/10.1111/oik.01719}

\bibitem[\citeproctext]{ref-poisotDescribeUnderstandPredict2016}
Poisot, T., Stouffer, D. B., \& Kéfi, S. (2016). Describe, understand
and predict: Why do we need networks in ecology? \emph{Functional
Ecology}, \emph{30}(12), 1878--1882.
\url{https://www.jstor.org/stable/48582345}

\bibitem[\citeproctext]{ref-pollockUnderstandingCooccurrenceModelling2014}
Pollock, L. J., Tingley, R., Morris, W. K., Golding, N., O'Hara, R. B.,
Parris, K. M., Vesk, P. A., \& McCarthy, M. A. (2014). Understanding
co-occurrence by modelling species simultaneously with a {Joint Species
Distribution Model} ({JSDM}). \emph{Methods in Ecology and Evolution},
\emph{5}(5), 397--406. \url{https://doi.org/10.1111/2041-210X.12180}

\bibitem[\citeproctext]{ref-pringleUntanglingFoodWebs2020}
Pringle, R. M. (2020). Untangling {Food Webs}. In \emph{Unsolved
{Problems} in {Ecology}} (pp. 225--238). Princeton University Press.
\url{https://doi.org/10.1515/9780691195322-020}

\bibitem[\citeproctext]{ref-proulxNetworkThinkingEcology2005}
Proulx, S. R., Promislow, D. E. L., \& Phillips, P. C. (2005). Network
thinking in ecology and evolution. \emph{Trends in Ecology \&
Evolution}, \emph{20}(6), 345--353.
\url{https://doi.org/10.1016/j.tree.2005.04.004}

\bibitem[\citeproctext]{ref-pykeOptimalForagingTheory1984}
Pyke, G. (1984). Optimal {Foraging Theory}: {A Critical Review}.
\emph{Annual Review of Ecology, Evolution and Systematic}, \emph{15},
523--575. \url{https://doi.org/10.1146/annurev.ecolsys.15.1.523}

\bibitem[\citeproctext]{ref-rooneyLandscapeTheoryFood2008}
Rooney, N., McCann, K. S., \& Moore, J. C. (2008). A landscape theory
for food web architecture. \emph{Ecology Letters}, \emph{11}(8),
867--881. \url{https://doi.org/10.1111/j.1461-0248.2008.01193.x}

\bibitem[\citeproctext]{ref-roopnarineExtinctionCascadesCatastrophe2006}
Roopnarine, P. D. (2006). Extinction {Cascades} and {Catastrophe} in
{Ancient Food Webs}. \emph{Paleobiology}, \emph{32}(1), 1--19.
\url{https://www.jstor.org/stable/4096814}

\bibitem[\citeproctext]{ref-saraviaEcologicalNetworkAssembly2022}
Saravia, L. A., Marina, T. I., Kristensen, N. P., De Troch, M., \& Momo,
F. R. (2022). Ecological network assembly: {How} the regional metaweb
influences local food webs. \emph{Journal of Animal Ecology},
\emph{91}(3), 630--642. \url{https://doi.org/10.1111/1365-2656.13652}

\bibitem[\citeproctext]{ref-segarRoleEvolutionShaping2020}
Segar, S. T., Fayle, T. M., Srivastava, D. S., Lewinsohn, T. M., Lewis,
O. T., Novotny, V., Kitching, R. L., \& Maunsell, S. C. (2020). The
{Role} of {Evolution} in {Shaping Ecological Networks}. \emph{Trends in
Ecology \& Evolution}, \emph{35}(5), 454--466.
\url{https://doi.org/10.1016/j.tree.2020.01.004}

\bibitem[\citeproctext]{ref-shawFrameworkReconstructingAncient2024}
Shaw, J. O., Dunhill, A. M., Beckerman, A. P., Dunne, J. A., \& Hull, P.
M. (2024). \emph{A framework for reconstructing ancient food webs using
functional trait data} (p. 2024.01.30.578036). bioRxiv.
\url{https://doi.org/10.1101/2024.01.30.578036}

\bibitem[\citeproctext]{ref-soberonGrinnellianEltonianNiches2007}
Soberón, J. (2007). Grinnellian and {Eltonian} niches and geographic
distributions of species. \emph{Ecology Letters}, \emph{10}(12),
1115--1123. \url{https://doi.org/10.1111/j.1461-0248.2007.01107.x}

\bibitem[\citeproctext]{ref-songRigorousValidationEcological2024}
Song, C., \& Levine, J. M. (2024). \emph{Rigorous (in)validation of
ecological models} (p. 2024.09.19.613075). bioRxiv.
\url{https://doi.org/10.1101/2024.09.19.613075}

\bibitem[\citeproctext]{ref-staniczenkoStructuralDynamicsRobustness2010}
Staniczenko, P. P. A., Lewis, O. T., Jones, N. S., \& Reed-Tsochas, F.
(2010). Structural dynamics and robustness of food webs. \emph{Ecology
Letters}, \emph{13}(7), 891--899.
\url{https://doi.org/10.1111/j.1461-0248.2010.01485.x}

\bibitem[\citeproctext]{ref-stockPairwiseLearningPredicting2021}
Stock, M. (2021). Pairwise learning for predicting pollination
interactions based on traits and phylogeny. \emph{Ecological Modelling},
14.

\bibitem[\citeproctext]{ref-stoufferAllEcologicalModels2019}
Stouffer, D. B. (2019). All ecological models are wrong, but some are
useful. \emph{Journal of Animal Ecology}, \emph{88}(2), 192--195.
\url{https://doi.org/10.1111/1365-2656.12949}

\bibitem[\citeproctext]{ref-strydomFoodWebReconstruction2022}
Strydom, T., Bouskila, S., Banville, F., Barros, C., Caron, D., Farrell,
M. J., Fortin, M.-J., Hemming, V., Mercier, B., Pollock, L. J., Runghen,
R., Dalla Riva, G. V., \& Poisot, T. (2022). Food web reconstruction
through phylogenetic transfer of low-rank network representation.
\emph{Methods in Ecology and Evolution}, \emph{13}(12), 2838--2849.
\url{https://doi.org/10.1111/2041-210X.13835}

\bibitem[\citeproctext]{ref-strydomGraphEmbeddingTransfer2023}
Strydom, T., Bouskila, S., Banville, F., Barros, C., Caron, D., Farrell,
M. J., Fortin, M.-J., Mercier, B., Pollock, L. J., Runghen, R., Dalla
Riva, G. V., \& Poisot, T. (2023). Graph embedding and transfer learning
can help predict potential species interaction networks despite data
limitations. \emph{Methods in Ecology and Evolution}, \emph{14}(12),
2917--2930. \url{https://doi.org/10.1111/2041-210X.14228}

\bibitem[\citeproctext]{ref-strydomRoadmapPredictingSpecies2021}
Strydom, T., Catchen, M. D., Banville, F., Caron, D., Dansereau, G.,
Desjardins-Proulx, P., Forero-Muñoz, N. R., Higino, G., Mercier, B.,
Gonzalez, A., Gravel, D., Pollock, L., \& Poisot, T. (2021). A roadmap
towards predicting species interaction networks (across space and time).
\emph{Philosophical Transactions of the Royal Society B: Biological
Sciences}, \emph{376}(1837), 20210063.
\url{https://doi.org/10.1098/rstb.2021.0063}

\bibitem[\citeproctext]{ref-terryFindingMissingLinks2020}
Terry, J. C. D., \& Lewis, O. T. (2020). Finding missing links in
interaction networks. \emph{Ecology}, \emph{101}(7), e03047.
\url{https://doi.org/10.1002/ecy.3047}

\bibitem[\citeproctext]{ref-valdovinosBioenergeticFrameworkAboveground2023}
Valdovinos, F. S., Hale, K. R. S., Dritz, S., Glaum, P. R., McCann, K.
S., Simon, S. M., Thébault, E., Wetzel, W. C., Wootton, K. L., \&
Yeakel, J. D. (2023). A bioenergetic framework for aboveground
terrestrial food webs. \emph{Trends in Ecology \& Evolution},
\emph{38}(3), 301--312. \url{https://doi.org/10.1016/j.tree.2022.11.004}

\bibitem[\citeproctext]{ref-vandewalleArthropodFoodWebs2023}
Van De Walle, R., Logghe, G., Haas, N., Massol, F., Vandegehuchte, M.
L., \& Bonte, D. (2023). Arthropod food webs predicted from body size
ratios are improved by incorporating prey defensive properties.
\emph{Journal of Animal Ecology}, \emph{92}(4), 913--924.
\url{https://doi.org/10.1111/1365-2656.13905}

\bibitem[\citeproctext]{ref-wellsSpeciesInteractionsEstimating2013}
Wells, K., \& O'Hara, R. B. (2013). Species interactions: Estimating
per-individual interaction strength and covariates before simplifying
data into per-species ecological networks. \emph{Methods in Ecology and
Evolution}, \emph{4}(1), 1--8.
\url{https://doi.org/10.1111/j.2041-210x.2012.00249.x}

\bibitem[\citeproctext]{ref-williamsSimpleRulesYield2000}
Williams, R. J., \& Martinez, N. D. (2000). Simple rules yield complex
food webs. \emph{Nature}, \emph{404}(6774), 180--183.
\url{https://doi.org/10.1038/35004572}

\bibitem[\citeproctext]{ref-williamsSuccessItsLimits2008}
Williams, R. J., \& Martinez, N. D. (2008). Success and its limits among
structural models of complex food webs. \emph{Journal of Animal
Ecology}, \emph{77}(3), 512--519.
\url{https://doi.org/10.1111/j.1365-2656.2008.01362.x}

\bibitem[\citeproctext]{ref-woottonModularTheoryTrophic2023}
Wootton, K. L., Curtsdotter, A., Roslin, T., Bommarco, R., \& Jonsson,
T. (2023). Towards a modular theory of trophic interactions.
\emph{Functional Ecology}, \emph{37}(1), 26--43.
\url{https://doi.org/10.1111/1365-2435.13954}

\bibitem[\citeproctext]{ref-xieCompletenessCommunityStructure2017}
Xie, J.-R., Zhang, P., Zhang, H.-F., \& Wang, B.-H. (2017). Completeness
of {Community Structure} in {Networks}. \emph{Scientific Reports},
\emph{7}(1), 5269. \url{https://doi.org/10.1038/s41598-017-05585-6}

\bibitem[\citeproctext]{ref-yeakelCollapseEcologicalNetwork2014}
Yeakel, J. D., Pires, M. M., Rudolf, L., Dominy, N. J., Koch, P. L.,
Guimarães, P. R., \& Gross, T. (2014). Collapse of an ecological network
in {Ancient Egypt}. \emph{PNAS}, \emph{111}(40), 14472--14477.
\url{https://doi.org/10.1073/pnas.1408471111}

\bibitem[\citeproctext]{ref-yodzisCompartmentationRealAssembled1982}
Yodzis, P. (1982). The {Compartmentation} of {Real} and {Assembled
Ecosystems}. \emph{The American Naturalist}, \emph{120}(5), 551--570.
\url{https://doi.org/10.1086/284013}

\bibitem[\citeproctext]{ref-yodzisBodySizeConsumerResource1992}
Yodzis, P., \& Innes, S. (1992). Body {Size} and {Consumer-Resource
Dynamics}. \emph{The American Naturalist}, \emph{139}(6), 1151--1175.
\url{https://doi.org/10.1086/285380}

\end{CSLReferences}




\end{document}
