% Options for packages loaded elsewhere
\PassOptionsToPackage{unicode}{hyperref}
\PassOptionsToPackage{hyphens}{url}
\PassOptionsToPackage{dvipsnames,svgnames,x11names}{xcolor}
%
\documentclass[
]{agujournal2019}

\usepackage{amsmath,amssymb}
\usepackage{iftex}
\ifPDFTeX
  \usepackage[T1]{fontenc}
  \usepackage[utf8]{inputenc}
  \usepackage{textcomp} % provide euro and other symbols
\else % if luatex or xetex
  \usepackage{unicode-math}
  \defaultfontfeatures{Scale=MatchLowercase}
  \defaultfontfeatures[\rmfamily]{Ligatures=TeX,Scale=1}
\fi
\usepackage{lmodern}
\ifPDFTeX\else  
    % xetex/luatex font selection
\fi
% Use upquote if available, for straight quotes in verbatim environments
\IfFileExists{upquote.sty}{\usepackage{upquote}}{}
\IfFileExists{microtype.sty}{% use microtype if available
  \usepackage[]{microtype}
  \UseMicrotypeSet[protrusion]{basicmath} % disable protrusion for tt fonts
}{}
\makeatletter
\@ifundefined{KOMAClassName}{% if non-KOMA class
  \IfFileExists{parskip.sty}{%
    \usepackage{parskip}
  }{% else
    \setlength{\parindent}{0pt}
    \setlength{\parskip}{6pt plus 2pt minus 1pt}}
}{% if KOMA class
  \KOMAoptions{parskip=half}}
\makeatother
\usepackage{xcolor}
\setlength{\emergencystretch}{3em} % prevent overfull lines
\setcounter{secnumdepth}{5}
% Make \paragraph and \subparagraph free-standing
\ifx\paragraph\undefined\else
  \let\oldparagraph\paragraph
  \renewcommand{\paragraph}[1]{\oldparagraph{#1}\mbox{}}
\fi
\ifx\subparagraph\undefined\else
  \let\oldsubparagraph\subparagraph
  \renewcommand{\subparagraph}[1]{\oldsubparagraph{#1}\mbox{}}
\fi


\providecommand{\tightlist}{%
  \setlength{\itemsep}{0pt}\setlength{\parskip}{0pt}}\usepackage{longtable,booktabs,array}
\usepackage{calc} % for calculating minipage widths
% Correct order of tables after \paragraph or \subparagraph
\usepackage{etoolbox}
\makeatletter
\patchcmd\longtable{\par}{\if@noskipsec\mbox{}\fi\par}{}{}
\makeatother
% Allow footnotes in longtable head/foot
\IfFileExists{footnotehyper.sty}{\usepackage{footnotehyper}}{\usepackage{footnote}}
\makesavenoteenv{longtable}
\usepackage{graphicx}
\makeatletter
\def\maxwidth{\ifdim\Gin@nat@width>\linewidth\linewidth\else\Gin@nat@width\fi}
\def\maxheight{\ifdim\Gin@nat@height>\textheight\textheight\else\Gin@nat@height\fi}
\makeatother
% Scale images if necessary, so that they will not overflow the page
% margins by default, and it is still possible to overwrite the defaults
% using explicit options in \includegraphics[width, height, ...]{}
\setkeys{Gin}{width=\maxwidth,height=\maxheight,keepaspectratio}
% Set default figure placement to htbp
\makeatletter
\def\fps@figure{htbp}
\makeatother
% definitions for citeproc citations
\NewDocumentCommand\citeproctext{}{}
\NewDocumentCommand\citeproc{mm}{%
  \begingroup\def\citeproctext{#2}\cite{#1}\endgroup}
\makeatletter
 % allow citations to break across lines
 \let\@cite@ofmt\@firstofone
 % avoid brackets around text for \cite:
 \def\@biblabel#1{}
 \def\@cite#1#2{{#1\if@tempswa , #2\fi}}
\makeatother
\newlength{\cslhangindent}
\setlength{\cslhangindent}{1.5em}
\newlength{\csllabelwidth}
\setlength{\csllabelwidth}{3em}
\newenvironment{CSLReferences}[2] % #1 hanging-indent, #2 entry-spacing
 {\begin{list}{}{%
  \setlength{\itemindent}{0pt}
  \setlength{\leftmargin}{0pt}
  \setlength{\parsep}{0pt}
  % turn on hanging indent if param 1 is 1
  \ifodd #1
   \setlength{\leftmargin}{\cslhangindent}
   \setlength{\itemindent}{-1\cslhangindent}
  \fi
  % set entry spacing
  \setlength{\itemsep}{#2\baselineskip}}}
 {\end{list}}
\usepackage{calc}
\newcommand{\CSLBlock}[1]{\hfill\break\parbox[t]{\linewidth}{\strut\ignorespaces#1\strut}}
\newcommand{\CSLLeftMargin}[1]{\parbox[t]{\csllabelwidth}{\strut#1\strut}}
\newcommand{\CSLRightInline}[1]{\parbox[t]{\linewidth - \csllabelwidth}{\strut#1\strut}}
\newcommand{\CSLIndent}[1]{\hspace{\cslhangindent}#1}

\usepackage{url} %this package should fix any errors with URLs in refs.
\usepackage{lineno}
\usepackage[inline]{trackchanges} %for better track changes. finalnew option will compile document with changes incorporated.
\usepackage{soul}
\linenumbers
\makeatletter
\@ifpackageloaded{caption}{}{\usepackage{caption}}
\AtBeginDocument{%
\ifdefined\contentsname
  \renewcommand*\contentsname{Table of contents}
\else
  \newcommand\contentsname{Table of contents}
\fi
\ifdefined\listfigurename
  \renewcommand*\listfigurename{List of Figures}
\else
  \newcommand\listfigurename{List of Figures}
\fi
\ifdefined\listtablename
  \renewcommand*\listtablename{List of Tables}
\else
  \newcommand\listtablename{List of Tables}
\fi
\ifdefined\figurename
  \renewcommand*\figurename{Figure}
\else
  \newcommand\figurename{Figure}
\fi
\ifdefined\tablename
  \renewcommand*\tablename{Table}
\else
  \newcommand\tablename{Table}
\fi
}
\@ifpackageloaded{float}{}{\usepackage{float}}
\floatstyle{ruled}
\@ifundefined{c@chapter}{\newfloat{codelisting}{h}{lop}}{\newfloat{codelisting}{h}{lop}[chapter]}
\floatname{codelisting}{Listing}
\newcommand*\listoflistings{\listof{codelisting}{List of Listings}}
\makeatother
\makeatletter
\makeatother
\makeatletter
\@ifpackageloaded{caption}{}{\usepackage{caption}}
\@ifpackageloaded{subcaption}{}{\usepackage{subcaption}}
\makeatother
\ifLuaTeX
  \usepackage{selnolig}  % disable illegal ligatures
\fi
\usepackage{bookmark}

\IfFileExists{xurl.sty}{\usepackage{xurl}}{} % add URL line breaks if available
\urlstyle{same} % disable monospaced font for URLs
\hypersetup{
  pdftitle={T is for Topology},
  pdfauthor={Tanya Strydom; Andrew P. Beckerman},
  pdfkeywords={food web, network construction},
  colorlinks=true,
  linkcolor={blue},
  filecolor={Maroon},
  citecolor={Blue},
  urlcolor={Blue},
  pdfcreator={LaTeX via pandoc}}

\journalname{Some fancy journal}

\draftfalse

\begin{document}
\title{T is for Topology}

\authors{Tanya Strydom\affil{1}, Andrew P. Beckerman\affil{1}}
\affiliation{1}{University of Sheffield, }
\correspondingauthor{Tanya Strydom}{t.strydom@sheffield.ac.uk}


\begin{abstract}
Pending\ldots{}
\end{abstract}

\section*{Plain Language Summary}
We want to know a bit more about the different network topology
generators (predict tools) and how they differ - \emph{i.e.,} their
strengths and weaknesses



\section{Introduction}\label{introduction}

The standard run of the mill that we cannot always feasibly construct
networks because 1. hard, 2. time (yay dinosaurs, but also the future
and impending doom I guess), and 3. probably something else meaningful
that's just slipping my mind at the moment. Some of the usual culprits
will come in here like: Jordano (2016b); (Jordano, 2016a); Poisot et al.
(2021); Strydom et al. (2021)

Maybe a brief history of the development of predictive tools/topo
generators? Sort of where the theory/body of work was based and how that
has changed? IS there a difference between toppo generator and
predictive tool - I'm inclined to think that it aligns with the whole
debate of high level structure vs node-level perfection

Maybe start here with discussing the core mechanistic differences that
models will work at --- some are really concerned about (and thus
constrained by) structure, others are more mechanistic in nature
\emph{i.e.,} species \emph{a} has the capacity to eat species \emph{b}
because traits (read gob size), and then you get Rohr et al. (2010) and
Strydom et al. (2022) that sit in the weird liminal latent space\ldots{}
Here I will probably get on my (newly discovered) soapbox and wax
lyrical about how in certain situations structure is enough (and that
will probably be for some high-level things like thinking about energy
flows etc., I can also see a world in which maybe you want to do some
sort of robustness/extinction work - since then you're usually doing
`random' (within limits) extinctions) but there may be use cases where
we are really interested in the node-level interactions \emph{i.e.,}
species identity is like a thing we need to care about and also be able
to retrieve specific interactions at specific nodes correctly.

At some point we are going to need to discuss the key differences and
implications between predicting a metaweb (\emph{sensu} Jennifer A.
Dunne (2006)) and a network realisation. And here I can't help but think
about Poisot et al. (2015) (and probably other papers) that discuss how
the local factors are going to play a role and even the same pair of
species may interact differently in different points in the landscape.

\begin{quote}
Do we need to delve into individual-based networks? (\emph{sensu} Tinker
2012, Araújo 2008) I think its probably a step too far and one starts
creeping into apples and pears type of comparisons. Especially since
these work off of already existing networks (I seem to recall) and its
more about about `tweaking' those - so not so much \emph{de novo}
predictions. Although this might be useful to keep in mind when it comes
to re-wiring\ldots{} Also on that note do we opn the re-wiring door here
in this ms or wait it out a bit.
\end{quote}

\section{Data \& Methods}\label{sec-data-methods}

\subsection{Overview of topology
generators}\label{overview-of-topology-generators}

I know table are awful but in this case they may make more sense. Not
sure about putting in some papers that have used the model - totes happy
to drop those I think\ldots{}

\begin{longtable}[]{@{}
  >{\raggedright\arraybackslash}p{(\columnwidth - 6\tabcolsep) * \real{0.1810}}
  >{\raggedright\arraybackslash}p{(\columnwidth - 6\tabcolsep) * \real{0.4381}}
  >{\raggedright\arraybackslash}p{(\columnwidth - 6\tabcolsep) * \real{0.1905}}
  >{\raggedright\arraybackslash}p{(\columnwidth - 6\tabcolsep) * \real{0.1905}}@{}}
\caption{Lets make a table that gives an overview of the different
topology generators that we will look
at}\label{tbl-history}\tabularnewline
\toprule\noalign{}
\begin{minipage}[b]{\linewidth}\raggedright
Approach
\end{minipage} & \begin{minipage}[b]{\linewidth}\raggedright
Reference
\end{minipage} & \begin{minipage}[b]{\linewidth}\raggedright
Core Mechanism
\end{minipage} & \begin{minipage}[b]{\linewidth}\raggedright
\emph{e.g.,} uses
\end{minipage} \\
\midrule\noalign{}
\endfirsthead
\toprule\noalign{}
\begin{minipage}[b]{\linewidth}\raggedright
Approach
\end{minipage} & \begin{minipage}[b]{\linewidth}\raggedright
Reference
\end{minipage} & \begin{minipage}[b]{\linewidth}\raggedright
Core Mechanism
\end{minipage} & \begin{minipage}[b]{\linewidth}\raggedright
\emph{e.g.,} uses
\end{minipage} \\
\midrule\noalign{}
\endhead
\bottomrule\noalign{}
\endlastfoot
Cascade model & Joel E. Cohen et al. (1990); Joel E. Cohen et al. (1997)
& structural & \\
Niche model & Williams \& Martinez (2000) & structural & \\
PFIM & Shaw et al. (2024) & mechanistic & Dunhill (in review) \\
Log-ratio & Rohr et al. (2010) & latent trait space & Yeakel et al.
(2014), Pires et al. (2020) (?) \\
Nested hierarchy & Cattin et al. (2004) & & \\
ADBM & Petchey et al. (2008) & mechanistic & probably multiple \\
Stochastic & Rossberg et al. (2006) & & \\
Graph Embedding & Strydom et al. (2023) & latent trait space & Strydom
et al. (2022) \\
Trait-based & Caron et al. (2022) & mechanistic & Caron et al. (2023) \\
\end{longtable}

\begin{quote}
Might be nice to have a little appendix/supp mat that breaks down the
models in detail so that they are all in one place so that someone (grad
student being told they need to build networks) some day can go and
educate themselves with slightly lower effort. This will also be useful
for me should I end up having to do some actual coding - think of this
as step one in the pseudo code process.
\end{quote}

\subsection{Datasets used}\label{datasets-used}

Here I think we need to span a variety of domains, at minimum aquatic
and terrestrial but maybe there should be a `scale' element as well
\emph{i.e.,} a regional and local network. I think there is going to be
a `turning point' where structural will take over from mechanistic in
terms of performance. More specifically at local scales bioenergetic
constraints (and co-occurrence) may play a bigger role in structuring a
network whereas at the metaweb level then mechanistic may make more
(since by default its about who can potentially interact and obviously
not constrained by real-world scenarios) \emph{sensu} Caron et al.
(2023). Although having said that I feel that contradicts the idea of
backbones (\emph{sensu} Bramon Mora (sp?) et al \& Stouffer et al) But
that might be where we get the idea of core \emph{structure} vs
something like linkage density. So core things like trophic level/chain
length will be conserved but connectance might not (I think I understand
what I'm trying to say here)

I think we should also use the Dunne (I think) Cambrian (also think)
network (I was correct and its this one Jennifer A. Dunne et al.
(2008)). Because 1) it gives the paleo-centric methods their moment in
the sun and 2) I think it also brings up the interesting question of can
we use modern structure to predict past ones? Here one might expect a
more mechanistic approach to shine.

\subsection{Comparing different
models}\label{comparing-different-models}

\begin{enumerate}
\def\labelenumi{\arabic{enumi}.}
\tightlist
\item
  Shortlist/finalise the different topo generators
\item
  collate/translate into \texttt{Julia}

  \begin{itemize}
  \tightlist
  \item
    \emph{e.g.,} some models wil be in ESpeciesInteractionNetworks.jl
    (new EcoNet); I know (parts of) the transfer learning stuff is and
    the niche model
  \item
    others will need to be coded out (the more simpler models should be
    easier)
  \item
    can also consider \texttt{R} but then it becomes a case of porting
    things left and right depending on how we decide to do the post
    analyses
  \end{itemize}
\item
  Create networks for the different datasets/scenarios we select - I
  feel like there might be some scenarios that we can't do all models
  for all datasets but maybe I'm being a pessimist.
\item
  compare model performance based on the ideas currently listed in the
  results section.
\item
  Make a pretty picture that summarises things - maybe overlapping Venn
  circles that showcase which models do well in the different
  spheres/aspects of life
\end{enumerate}

\section{Results}\label{results}

How we want to compare and contrast. I think there won't be a `winner'
and thus we need to think of `tests' that are going to measure
performance in different situations/settings. With that in mind I think
some valuable points to consider would be:

\begin{itemize}
\tightlist
\item
  Structural vs pairwise link predictions (graph vs node level)

  \begin{itemize}
  \tightlist
  \item
    \% of links correctly retrieved
  \item
    connectance
  \item
    trophic level
  \item
    generalism vs specialism
  \item
    something related to false positives/negatives
  \item
    intervality
  \end{itemize}
\item
  Data `cost' (some methods might need a lot lot of supporting data vs
  something very light weight)
\item
  I think it would be remiss to not also take into consideration
  computational cost
\item
  something about the network output - I'm acknowledging my biases and
  saying that probabilistic (or \emph{maybe} weighted) links are the way
\end{itemize}

\begin{quote}
maybe we can put these into broader categories - if we do start doing
the venn overlap thing. \emph{E.g.,} local scale predictions, regional
scale predictions, pairwise interactions, structural (energetics),
computationally cheap, low cost data
\end{quote}

\section{Discussion}\label{discussion}

I think a big take home will (hopefully) be how different approaches do
better in different situations and so you as an end user need to take
this into consideration and pick accordingly.

I probably think about this point too much but a point of discussion
that I think will be interesting to bring up the idea that if a model is
missing a specific pairwise link but doing well at the structural level
when does it matter? I think this is covered with the whole node vs
graph level performance but I kind of just want to bring it up here
again because also one of those things that I think about a bit too much
probably\ldots{}

\begin{quote}
Thinking very long term here and maybe a bit beyond the scope but also
thinking about a multi- model approach? SO using one model to build an
initial network but maybe a second one to constrain it a bit better. I
blame this thought on the over-connected PFIM food webs\ldots{}
\end{quote}

\section*{References}\label{references}
\addcontentsline{toc}{section}{References}

\vspace{1em}

\textsubscript{Source:
\href{https://BecksLab.github.io/ms_t_is_for_topology/index.qmd.html}{Article
Notebook}}

\phantomsection\label{refs}
\begin{CSLReferences}{1}{0}
\bibitem[\citeproctext]{ref-caronAddressingEltonianShortfall2022}
Caron, D., Maiorano, L., Thuiller, W., \& Pollock, L. J. (2022).
Addressing the {Eltonian} shortfall with trait-based interaction models.
\emph{Ecology Letters}, \emph{25}(4), 889--899.
\url{https://doi.org/10.1111/ele.13966}

\bibitem[\citeproctext]{ref-caronTrophicInteractionModels2023}
Caron, D., Brose, U., Lurgi, M., Blanchet, G., Gravel, D., \& Pollock,
L. J. (2023, May). Trophic interaction models predict interactions
across space, not food webs. {EcoEvoRxiv}.
\url{https://doi.org/10.32942/X29K55}

\bibitem[\citeproctext]{ref-cattinPhylogeneticConstraintsAdaptation2004}
Cattin, M.-F., Bersier, L.-F., Banašek-Richter, C., Baltensperger, R.,
\& Gabriel, J.-P. (2004). Phylogenetic constraints and adaptation
explain food-web structure. \emph{Nature}, \emph{427}(6977), 835--839.
\url{https://doi.org/10.1038/nature02327}

\bibitem[\citeproctext]{ref-cohenCommunityFoodWebs1990}
Cohen, Joel E., Briand, F., \& Newman, C. (1990). \emph{Community {Food
Webs}: {Data} and {Theory}}. {Berlin Heidelberg}: {Springer-Verlag}.

\bibitem[\citeproctext]{ref-cohenStochasticTheoryCommunity1997}
Cohen, Joel E., Newman, C. M., \& Steele, J. H. (1997). A stochastic
theory of community food webs {I}. {Models} and aggregated data.
\emph{Proceedings of the Royal Society of London. Series B. Biological
Sciences}, \emph{224}(1237), 421--448.
\url{https://doi.org/10.1098/rspb.1985.0042}

\bibitem[\citeproctext]{ref-dunneNetworkStructureFood2006}
Dunne, Jennifer A. (2006). The {Network Structure} of {Food Webs}. In
Jennifer A. Dunne \& M. Pascual (Eds.), \emph{Ecological networks:
{Linking} structure and dynamics} (pp. 27--86). {Oxford University
Press}.

\bibitem[\citeproctext]{ref-dunneCompilationNetworkAnalyses2008}
Dunne, Jennifer A., Williams, R. J., Martinez, N. D., Wood, R. A., \&
Erwin, D. H. (2008). Compilation and {Network Analyses} of {Cambrian
Food Webs}. \emph{PLOS Biology}, \emph{6}(4), e102.
\url{https://doi.org/10.1371/journal.pbio.0060102}

\bibitem[\citeproctext]{ref-jordanoChasingEcologicalInteractions2016}
Jordano, P. (2016a). Chasing {Ecological Interactions}. \emph{PLOS
Biology}, \emph{14}(9), e1002559.
\url{https://doi.org/10.1371/journal.pbio.1002559}

\bibitem[\citeproctext]{ref-jordanoSamplingNetworksEcological2016}
Jordano, P. (2016b). Sampling networks of ecological interactions.
\emph{Functional Ecology}. \url{https://doi.org/10.1111/1365-2435.12763}

\bibitem[\citeproctext]{ref-petcheySizeForagingFood2008}
Petchey, O. L., Beckerman, A. P., Riede, J. O., \& Warren, P. H. (2008).
Size, foraging, and food web structure. \emph{Proceedings of the
National Academy of Sciences}, \emph{105}(11), 4191--4196.
\url{https://doi.org/10.1073/pnas.0710672105}

\bibitem[\citeproctext]{ref-piresMegafaunalExtinctionsHuman2020}
Pires, M. M., Rindel, D., Moscardi, B., Cruz, L. R., Guimarães, P. R.,
dos Reis, S. F., \& Perez, S. I. (2020). Before, during and after
megafaunal extinctions: {Human} impact on {Pleistocene-Holocene} trophic
networks in {South Patagonia}. \emph{Quaternary Science Reviews},
\emph{250}, 106696.
\url{https://doi.org/10.1016/j.quascirev.2020.106696}

\bibitem[\citeproctext]{ref-poisotSpeciesWhyEcological2015}
Poisot, T., Stouffer, D. B., \& Gravel, D. (2015). Beyond species: Why
ecological interaction networks vary through space and time.
\emph{Oikos}, \emph{124}(3), 243--251.
\url{https://doi.org/10.1111/oik.01719}

\bibitem[\citeproctext]{ref-poisotGlobalKnowledgeGaps2021}
Poisot, T., Bergeron, G., Cazelles, K., Dallas, T., Gravel, D.,
MacDonald, A., et al. (2021). Global knowledge gaps in species
interaction networks data. \emph{Journal of Biogeography},
\emph{n/a}(n/a). \url{https://doi.org/10.1111/jbi.14127}

\bibitem[\citeproctext]{ref-rohrModelingFoodWebs2010}
Rohr, R. P., Scherer, H., Kehrli, P., Mazza, C., \& Bersier, L.-F.
(2010). Modeling {Food Webs}: {Exploring Unexplained Structure Using
Latent Traits}. \emph{The American Naturalist}, \emph{176}(2), 170--177.
\url{https://doi.org/10.1086/653667}

\bibitem[\citeproctext]{ref-rossbergFoodWebsExperts2006}
Rossberg, A. G., Matsuda, H., Amemiya, T., \& Itoh, K. (2006). Food
webs: {Experts} consuming families of experts. \emph{Journal of
Theoretical Biology}, \emph{241}(3), 552--563.
\url{https://doi.org/10.1016/j.jtbi.2005.12.021}

\bibitem[\citeproctext]{ref-shawFrameworkReconstructingAncient2024}
Shaw, J. O., Dunhill, A. M., Beckerman, A. P., Dunne, J. A., \& Hull, P.
M. (2024, January). A framework for reconstructing ancient food webs
using functional trait data. {bioRxiv}.
\url{https://doi.org/10.1101/2024.01.30.578036}

\bibitem[\citeproctext]{ref-strydomRoadmapPredictingSpecies2021}
Strydom, T., Catchen, M. D., Banville, F., Caron, D., Dansereau, G.,
Desjardins-Proulx, P., et al. (2021). A roadmap towards predicting
species interaction networks (across space and time).
\emph{Philosophical Transactions of the Royal Society B: Biological
Sciences}, \emph{376}(1837), 20210063.
\url{https://doi.org/10.1098/rstb.2021.0063}

\bibitem[\citeproctext]{ref-strydomFoodWebReconstruction2022}
Strydom, T., Bouskila, S., Banville, F., Barros, C., Caron, D., Farrell,
M. J., et al. (2022). Food web reconstruction through phylogenetic
transfer of low-rank network representation. \emph{Methods in Ecology
and Evolution}, \emph{13}(12), 2838--2849.
\url{https://doi.org/10.1111/2041-210X.13835}

\bibitem[\citeproctext]{ref-strydomGraphEmbeddingTransfer2023}
Strydom, T., Bouskila, S., Banville, F., Barros, C., Caron, D., Farrell,
M. J., et al. (2023). Graph embedding and transfer learning can help
predict potential species interaction networks despite data limitations.
\emph{Methods in Ecology and Evolution}, \emph{14}(12), 2917--2930.
\url{https://doi.org/10.1111/2041-210X.14228}

\bibitem[\citeproctext]{ref-williamsSimpleRulesYield2000}
Williams, R. J., \& Martinez, N. D. (2000). Simple rules yield complex
food webs. \emph{Nature}, \emph{404}(6774), 180--183.
\url{https://doi.org/10.1038/35004572}

\bibitem[\citeproctext]{ref-yeakelCollapseEcologicalNetwork2014}
Yeakel, J. D., Pires, M. M., Rudolf, L., Dominy, N. J., Koch, P. L.,
Guimarães, P. R., \& Gross, T. (2014). Collapse of an ecological network
in {Ancient Egypt}. \emph{PNAS}, \emph{111}(40), 14472--14477.
\url{https://doi.org/10.1073/pnas.1408471111}

\end{CSLReferences}



\end{document}
